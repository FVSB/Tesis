\chapter*{Resumen}
El problema de optimización bi-nivel se define como minimizar una función sobre un conjunto definido como puntos óptimos de un modelo de programación matemática. La optimización en el nivel inferior depende de las decisiones tomadas en el nivel superior, y viceversa, creando así una relación de interdependencia entre los dos niveles.
Para ello se considera una relajación en la cual el problema del nivel inferior se sustituye por las condiciones necesarias de optimalidad. Aun así es un problema complejo pues no cumple con condiciones de regularidad. Los algoritmos que le dan solución son complejos de ahí es importante medir su eficiencia, en particular hallar al menos puntos que satisfagan las condiciones necesarias de optimalidad.
En este trabajo se propone una forma de generar problemas de dos niveles cuyo problema relajado tiene un punto estacionario de una de las clases: strongly stationary, M-stationary o C en dependencia de los multiplicadores.
Para ello se construyen las funciones usando perturbaciones lineales o cuadráticas que definen un problema de dos niveles para el cual, el punto dado es un punto estacionario del problema relajado. Los modelos asi generados se resuelven en la biblioteca BilevelJuMP de Julia. Se utilizan criterios como la evaluacion de la funcion objetivo y la factiblidad para comparar la calidad de la solución  calculada con la del punto que se sabe es estacionario.

\textbf{Palabras clave:} Optimización bi-nivel, MPECs, Condiciones necesarias de optimalidad, Punto estacionario.

\chapter*{Abstract}
The bi-level optimization problem is defined as minimizing a function over a set defined as the optimal points of a mathematical programming model. The optimization at the lower level depends on the decisions made at the upper level, and vice versa, thus creating an interdependent relationship between the two levels.
For this, a relaxation is considered in which the lower level problem is replaced by the necessary optimality conditions. Even so, it is a complex problem as it does not meet regularity conditions. The algorithms that solve it are complex, hence it is important to measure their efficiency, in particular to find at least points that satisfy the necessary optimality conditions.
This work proposes a way to generate bi-level problems whose relaxed problem has a stationary point of one of the classes: strongly stationary, M-stationary, or C-stationary depending on the multipliers.
To do this, functions are constructed using linear or quadratic perturbations that define a bi-level problem for which the given point is a stationary point of the relaxed problem. The models thus generated are solved in the BilevelJuMP library of Julia. Criteria such as the evaluation of the objective function and feasibility are used to compare the quality of the solution calculated with that of the point known to be stationary.

\textbf{Keywords:} Bi-level optimization, MPECs, Necessary optimality conditions, Stationary point.
