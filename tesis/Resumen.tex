\chapter*{Resumen}
El problema de optimización bi-nivel se define como minimizar una función sobre un conjunto definido como puntos óptimos de un modelo de programación matemática. La optimización en el nivel inferior depende de las decisiones tomadas en el nivel superior, y viceversa, creando así una relación de interdependencia entre los dos niveles.
Resolver este tipo de modelos es costoso dado que es un problema NP hallar un punto factible y con mayor complejidad un óptimo. Para ello se considera una relajación en la cual el problema del nivel inferior se sustituye por las condiciones necesarias de optimalidad. Aun así es un problema complejo pues no cumple con condiciones de regularidad. Los algoritmos que le dan solución son complejos de ahí es importante medir su eficiencia, en particular hallar al menos puntos que satisfagan las condiciones necesarias de optimalidad. Un criterio importante es la calidad de la solución, que se puede medir comparando el valor de la función objetivo con alguna solución conocida o si cumple una condición  de optimalidad. En este trabajo  se propone una forma de generar problemas de dos niveles con un punto estacionario conocido. Luego de estudiar las condiciones de optimalidad y algoritmos  para problemas bi-niveles, se construyen las funciones usando perturbaciones lineales o cuadráticas y se prueba la calidad de algoritmos implementados en problemas así 
generados\\
\textbf{Palabras clave:} Optimización bi-nivel, Función objetivo, Modelo de programación matemática, Interdependencia, Problema NP, Relajación, Condiciones necesarias de optimalidad, Complejidad, Eficiencia de algoritmos, Calidad de la solución, Punto estacionario, Perturbaciones lineales y cuadráticas, Algoritmos implementados

\chapter*{Abstract}
The bi-level optimization problem is defined as minimizing a function over a set defined as optimal points of a mathematical programming model. The optimization at the lower level depends on the decisions made at the upper level, and vice versa, thus creating an interdependent relationship between the two levels.
Solving this type of model is costly since it is an NP problem to find a feasible point and even more complex to find an optimal one. To address this, a relaxation is considered in which the lower-level problem is replaced by the necessary optimality conditions. Nevertheless, it remains a complex problem as it does not satisfy regularity conditions. The algorithms that provide solutions are complex; therefore, it is important to measure their efficiency, particularly in finding at least points that satisfy the necessary optimality conditions. An important criterion is the quality of the solution, which can be measured by comparing the value of the objective function with some known solution or by checking if it meets an optimality condition.
In this work, a method is proposed to generate two-level problems with a known stationary point. After studying the optimality conditions and algorithms for bi-level problems, functions are constructed using linear or quadratic perturbations, and the quality of implemented algorithms is tested on problems generated in this 
way\\
\textbf{Keywords:} Bi-level optimization, Objective function, Mathematical programming model, Interdependence, NP problem, Relaxation, Necessary optimality conditions, Complexity, Algorithm efficiency, Solution quality, Stationary point, Linear and quadratic perturbations, Implemented algorithms
