\begin{conclusions}
    

    %Explicar que se hizo
    En esta tesis se desarrolló un algoritmo para la generación de problemas de optimización binivel con características específicas de estacionariedad en puntos determinados. El mismo tiene la capacidad de modificar problemas originales para garantizar la factibilidad y estacionariedad en puntos dados, aprovechando las capacidades del lenguaje de programación Julia, que destaca por su alto rendimiento computacional y sintaxis adaptada a problemas de optimización.
    
    % Explicar como fue la experimentación
    La experimentación se llevó a cabo sobre tres categorías fundamentales de problemas: lineales, cuadráticos y no convexos. El proceso experimental comenzó con la obtención de puntos mínimos utilizando bibliotecas establecidas como BilevelJuMP y JuMP. Posteriormente, se generaron problemas estacionarios mediante la adición de componentes aleatorias a estos puntos para cada clase de problema definida.
    
    % Que se comparó
    Los problemas modificados fueron sometidos nuevamente a algoritmos tradicionales implementados en Julia para evaluar su efectividad frente a los puntos estacionarios generados. Se realizó un análisis comparativo destacando los casos más relevantes de cada categoría de puntos estacionarios, considerando la evaluación de la función objetivo del nivel superior en el punto donde se garantizó la estacionariedad, contrastándola con los resultados obtenidos por las bibliotecas convencionales.
    
% Tabla principal con subtablas
\begin{resultstablee}{Resultados de los problemas}

    \multicolumn{9}{|c|}{\textbf{Alpha-Zero}} \\
    \hline
    \resultroww{Tipo A}{Problema 1}{(1, 2)}{5.0}{(3, 4)}{10.0}{Método X}{Factible}{Convergencia}
    \resultroww{Tipo B}{Problema 2}{(5, 6)}{15.0}{(7, 8)}{20.0}{Método Y}{No factible}{Iteraciones máximas}
    
    \multicolumn{9}{|c|}{\textbf{C-Estacionario}} \\
    \hline
    \resultroww{Tipo C}{Problema 3}{(2, 3)}{7.0}{(4, 5)}{12.0}{Método Z}{Factible}{Convergencia}
    \resultroww{Tipo D}{Problema 4}{(6, 7)}{18.0}{(8, 9)}{25.0}{Método W}{No factible}{Error numérico}
    
    \multicolumn{9}{|c|}{\textbf{M-Estacionario}} \\
    \hline
    \resultroww{Tipo E}{Problema 5}{(3, 4)}{9.0}{(5, 6)}{14.0}{Método V}{Factible}{Convergencia}
    \resultroww{Tipo F}{Problema 6}{(7, 8)}{20.0}{(9, 10)}{30.0}{Método U}{No factible}{Iteraciones máximas}
    
    \multicolumn{9}{|c|}{\textbf{Fuertemente-Estacionario}} \\
    \hline
    \resultroww{Tipo G}{Problema 7}{(4, 5)}{11.0}{(6, 7)}{16.0}{Método T}{Factible}{Convergencia}
    \resultroww{Tipo H}{Problema 8}{(8, 9)}{22.0}{(10, 11)}{35.0}{Método S}{No factible}{Error numérico}
\end{resultstablee}
            
        
\end{conclusions}
