\chapter{Introducción}
% Primer parráfo introductorio 
La optimización binivel es una herramienta matemática que permite modelar situaciones complejas donde intervienen dos niveles de decisión jerárquicos. Este enfoque se utiliza en diversas áreas, desde la planificación de redes de distribución hasta la gestión de recursos en entornos industriales. La importancia de estos problemas radica en su capacidad para reflejar decisiones interdependientes, donde un decisor superior (o líder) establece un marco dentro del cual un decisor inferior (o seguidor) toma decisiones que afectan el resultado global.

La optimización binivel es un problema de optimización en el cual un subconjunto de variables está restringido a ser la solución óptima de otro problema de optimización, el cual está parametrizado por las variables restantes. Este tipo de problema tiene dos niveles jerárquicos de decisión: el problema de nivel superior o del líder, y el problema de nivel inferior o del seguidor. 

En términos abstractos, la optimización binivel busca minimizar una función objetivo de nivel superior, $F(x, y)$, donde $x$ son las variables de decisión del líder y $y$ son las variables del seguidor. Esta minimización está sujeta a dos tipos de restricciones: las restricciones explícitas para el líder, $x \in X$, donde $X$ es el conjunto de valores factibles para las variables del líder; y las restricciones implícitas impuestas por el seguidor, donde $y$ debe pertenecer al conjunto de soluciones óptimas del problema de optimización del seguidor, $\arg\min\{f(x, y) : y \in Y(x)\}$. En este contexto, $f(x, y)$ es la función objetivo del nivel inferior, y $Y(x)$ representa las restricciones del nivel inferior, las cuales pueden depender de las variables de decisión del líder, $x$.

En otras palabras, el problema de optimización binivel se centra en que el líder (nivel superior) debe tomar decisiones ($x$) que optimicen su objetivo $F(x, y)$, anticipando que el seguidor (nivel inferior) responderá de manera óptima con respecto a su propio objetivo $f(x, y)$, dado el valor de $x$ elegido por el líder. Esta interacción jerárquica entre ambos niveles añade una gran complejidad al problema en comparación con los problemas de optimización de un solo nivel.

Un problema de optimización binivel tiene dos características principales: en primer lugar, el problema del nivel inferior actúa como una restricción para el problema del nivel superior, y en segundo lugar, la solución del nivel inferior depende del valor de las variables del nivel superior, creando una interdependencia entre ambos niveles. Por ello, el líder debe anticipar la respuesta óptima del seguidor al tomar sus decisiones.

La formulación general de un problema de optimización binivel se expresa matemáticamente como:  

% Definición de problema binivel
\begin{equation}
\begin{aligned}
\text{minimizar} & \quad F(x, y) \\
\text{sujeto a} & \quad G(x, y) \leq 0 \quad (\text{restricciones de desigualdad}) \\
& \quad H(x, y) = 0 \quad (\text{restricciones de igualdad}) \\
& \quad y \in S(x) = \arg \min_{y} \{ f(x, y) \mid g(x, y) \leq 0, h(x, y) = 0 \}.
\end{aligned}
\end{equation}

Los elementos clave en la optimización binivel incluyen las funciones objetivo $F(x, y)$ y $f(x, y)$, que corresponden a los objetivos del líder y del seguidor, respectivamente; las restricciones $G(x, y) \leq 0$ y $H(x, y) = 0$, que deben ser satisfechas por ambas partes; y el conjunto de soluciones del seguidor $S(x)$, el cual representa las soluciones óptimas del nivel inferior en función de las decisiones del líder.
% Aplicaciones 

La optimización binivel se presenta como una herramienta fundamental para modelar y analizar los complejos mercados eléctricos, ofreciendo una perspectiva única sobre las interacciones estratégicas entre diversos agentes económicos.
% Mercado Eléctrico
% Nash equilibrium in a pay-as-bid electricity market Part 2 - best response of a producer Didier
En el trabajo de \cite{Aussel2017NashEI}, se profundiza en el análisis de los mercados de electricidad de pago por oferta, explorando cómo un productor puede ajustar su estrategia considerando las acciones de sus competidores. El estudio destaca la aplicación de conceptos de \textbf{equilibrio de Nash} y técnicas de mejor respuesta, proporcionando una metodología sofisticada para optimizar la participación de un productor en el mercado.
% Deregulated electricity markets with thermal losses and production bounds: models and optimality conditions
Continuando con esta línea de investigación, \cite{Aussel2016DeregulatedEM} desarrollaron un modelo innovador que aborda los mercados de electricidad desregulados. Su enfoque se distingue por incorporar restricciones de producción y pérdidas térmicas, lo que permite una modelización más precisa y realista. Mediante herramientas de \textbf{modelado de mercados y análisis de condiciones de optimalidad}, los investigadores pueden explorar escenarios más complejos y representativos del funcionamiento real de los mercados energéticos.
% Electricity spot market with transmission losses
Un trabajo posterior de \cite{Aussel2013ElectricitySM} introduce un elemento crítico en la modelización de mercados eléctricos: las pérdidas de transmisión. Esta contribución mejora significativamente la representación del sistema eléctrico, permitiendo un análisis más profundo del equilibrio estratégico mediante técnicas de \textbf{optimización de mercado}. Al considerar las pérdidas de transmisión, el modelo captura aspectos fundamentales de la distribución y comercialización de energía que anteriormente pasaban desapercibidos.


% Machine Learning
La optimización binivel también tiene aplicaciones fundamentales en la selección de hiperparámetros en aprendizaje automático, como lo demuestra el trabajo de \cite{DempeyZemkoho2020ML}. El capítulo 6 del libro aborda la optimización de hiperparámetros en problemas de clasificación y regresión. Se presentan algoritmos innovadores para manejar funciones objetivo no suaves y no convexas. La razón de uso de la optimización binivel radica en su capacidad para minimizar errores en modelos complejos, mejorando así la precisión general del aprendizaje automático. Además, se implementan algoritmos especializados para abordar problemas no convexos.
% EPI
La optimización binivel es una herramienta clave en el diseño y operación de redes industriales sostenibles. Ejemplos notables incluyen redes de agua industrial,
% Water integration in eco-industrial parks using a multi-leader-follower approach
donde en los estudios de \cite{Ramos2016WaterII} se optimizan redes de agua industrial mediante juegos de múltiples líderes-seguidores, priorizando objetivos ambientales y económicos; los resultados muestran que las empresas participantes lograron beneficios significativos en escenarios con formulaciones KKT.
%  Utility network optimization in eco-industrial parks by a multi-leader follower game methodology
Además, \cite{Ramos2018UtilityNO} introducen el concepto de autoridad ambiental en el diseño de redes de servicios públicos, utilizando juegos de múltiples líderes-seguidores y reformulaciones KKT.
% Bi-level optimal low-carbon economic dispatch for an industrial park with consideration of multi-energy price incentive
En el ámbito del despacho energético bajo restricciones de carbono, \cite{Gu2020BilevelOL} modela incentivos de precios de energía en un parque industrial, demostrando que un enfoque binivel puede simultáneamente mejorar el impacto ambiental y los beneficios económicos, utilizando un procedimiento iterativo primal-dual.

% Bilevel Díficiles
% NP-Hard
% The polynomial hierarchy and a simple model for competitive analysis
% Some properties of the bilevel programming problem
Dado que los problemas de optimización binivel son inherentemente difíciles de resolver debido a su naturaleza \textbf{NP-hard} \cite{Jeroslow1985ThePHNP,jonathan_f__bard_1991NP}
% Sigma P2Hard
% Libro Dempe
% Tesis doctoral Cerulli
o incluso $\Sigma P2-hard$ \cite{phdthesisCerulli,DempeyZemkoho2020},
se han desarrollado diversos enfoques para abordar su complejidad computacional. Entre los métodos exactos más utilizados se encuentran las reformulaciones basadas en las condiciones KKT (Karush-Kuhn-Tucker), que permiten transformar el problema binivel en un problema mononivel resoluble mediante técnicas tradicionales de programación matemática.
Sin embargo, estos enfoques suelen ser computacionalmente intensivos para problemas de gran escala \cite{phdthesisCerulli}.
%A Review on Bilevel Optimization: From Classical to Evolutionary Approaches and Applications
En paralelo, los algoritmos metaheurísticos, como los evolutivos, han ganado relevancia al proporcionar aproximaciones eficientes en casos no lineales o no convexos, donde las soluciones exactas son inalcanzables en tiempos razonables \cite{Sinha2017ARO}.
% Dividir en subproblemas y resolver iterativamente
% Collection of Test Problems for Constrained Global Optimization Algorithms
Otro enfoque destacado es el uso de métodos basados en descomposición, los cuales dividen el problema en subproblemas más manejables que pueden resolverse iterativamente \cite{Floudas1990ACO}.
Además, los avances recientes han explorado el uso de técnicas probabilísticas, como las aproximaciones de máxima entropía, para problemas con incertidumbre en parámetros clave.
% An SOS1-based approach for solving MPECs with a natural gas market application
Estas técnicas son particularmente útiles en aplicaciones prácticas, como los mercados de energía o los modelos de sostenibilidad \cite{SadddiquiNaturalGasSOS1}.
% Libro Dempe
A pesar de estos avances, existen desafíos abiertos. La escalabilidad sigue siendo un problema crítico, ya que el crecimiento exponencial de las opciones posibles en problemas de gran tamaño limita la aplicabilidad de los métodos exactos \cite{DempeyZemkoho2020}.
Asimismo, los problemas no convexos carecen de garantías de convergencia hacia el óptimo global, lo que los hace especialmente difíciles de abordar. Finalmente, la incorporación de incertidumbre en los modelos agrega una capa adicional de complejidad, lo que demanda nuevos enfoques híbridos que combinen algoritmos exactos y heurísticos para mejorar la eficiencia computacional sin sacrificar la calidad de las soluciones \cite{phdthesisCerulli,Sinha2017ARO}. Estos avances y desafíos reflejan la importancia de diseñar algoritmos personalizados que aprovechen las estructuras particulares de cada problema binivel. Las aplicaciones industriales, como el diseño de redes ecoindustriales o la gestión de mercados energéticos, destacan la necesidad de enfoques que equilibren precisión y tiempo de cálculo, haciendo de la optimización binivel un área de investigación activa y con un impacto significativo en la práctica.

% DIFERENCIA OPTIMISTA PESIMISTA
% Decir que en el libro de Dempe se hablan de estos dos enfoques
La optimización binivel es un campo de estudio que presenta dos enfoques principales: el optimista y el pesimista. En el enfoque optimista, se asume que el seguidor, que actúa en el nivel inferior, elegirá la solución más favorable para el líder, quien toma decisiones en el nivel superior. Este enfoque es considerado más tratable y, en ciertas situaciones favorables, puede simplificarse a un problema convexo. Además, en el contexto de múltiples objetivos, el enfoque optimista permite alcanzar el mejor frente de Pareto posible \cite{DempeyZemkoho2020}.

Por otro lado, el enfoque pesimista asume que el seguidor seleccionará la opción menos favorable para el líder entre las soluciones óptimas disponibles. Este enfoque es más complejo de resolver y puede incluso no tener solución. A menudo, se requieren reformulaciones para abordar estos problemas, lo que lo convierte en un reto teórico y computacional significativo. En situaciones de múltiples objetivos, el enfoque pesimista conduce al peor frente de Pareto posible \cite{Sinha2017ARO}.

Es relevante destacar que la mayoría de la literatura sobre optimización binivel se centra en el enfoque optimista debido a su mayor facilidad de tratamiento. Sin embargo, el enfoque pesimista también tiene su utilidad, especialmente en la modelación de situaciones donde se considera la aversión al riesgo 
\cite{DempeyZemkoho2020}. En este contexto, los términos "líder" y "seguidor" se utilizan para describir los roles en el problema de optimización; el líder toma decisiones considerando las posibles reacciones del seguidor, quien a su vez reacciona seleccionando su mejor opción \cite{Sinha2017ARO}.

% Sobre Valor extremal kkt y algoritmos 

El principio extremal es esencial para el análisis variacional y la derivación de condiciones de optimalidad. Este principio se utiliza para establecer reglas de cálculo y aplicaciones en la optimización, especialmente dentro del enfoque geométrico del análisis variacional. Aunque el concepto de valor extremal no siempre se nombra explícitamente, está intrínsecamente relacionado con el análisis variacional en problemas binivel, lo que implica que su comprensión es crucial para abordar estos problemas de manera efectiva \cite{DempeyZemkoho2020}.
Las condiciones KKT son una herramienta clave en la reformulación de problemas de optimización, particularmente cuando el problema de nivel inferior es convexo. Estas condiciones son necesarias y suficientes para garantizar la optimalidad en problemas de programación matemática, como se establece en la literatura sobre programación no lineal. Se destaca que las condiciones KKT permiten transformar problemas de optimización binivel en programas matemáticos con restricciones de equilibrio (MPEC), facilitando así su resolución \cite{DempeyZemkoho2020}.
Los algoritmos que se basan en las condiciones KKT incluyen una variedad de métodos, tales como técnicas de branch-and-bound y métodos de suavizado. Estos algoritmos son utilizados para resolver problemas complejos de optimización que involucran restricciones. Por ejemplo, se discute el uso de un método de suavizado junto con las condiciones KKT para abordar problemas relacionados con la optimización de hiperparámetros. Además, se menciona que los algoritmos SQP (Sequential Quadratic Programming) también se fundamentan en las condiciones KKT para resolver problemas suaves con restricciones. La versatilidad del método KKT se extiende incluso a su aplicación en algoritmos evolutivos, lo que demuestra su relevancia en diversas áreas de la optimización \cite{DempeyZemkoho2020}.

% Obtener una garantia de la solucion es muy complejo (generalmente es estacionario y con puntos estacionarios).
La optimización de dos niveles, un área fundamental en la investigación operativa y la teoría de juegos, presenta desafíos significativos debido a su complejidad inherente. Este tipo de problemas se caracteriza por la interacción entre un líder y un seguidor, donde las decisiones del líder afectan las respuestas del seguidor. Uno de los aspectos más críticos de esta problemática es garantizar la existencia de soluciones óptimas, lo cual se ve complicado por la naturaleza no convexa del problema, incluso cuando las funciones y los conjuntos factibles son convexos. A menudo, los algoritmos utilizados en este contexto solo logran identificar puntos estacionarios o críticos, que no necesariamente representan soluciones locales o globales \cite{DempeyZemkoho2020}.

Los problemas de optimización de dos niveles son inherentemente no convexos, lo que implica que los métodos de optimización convexa no son directamente aplicables. Esta no convexidad puede llevar a soluciones subóptimas y a dificultades para encontrar una solución global. A pesar de que en muchos casos las funciones y conjuntos factibles pueden ser convexos, la estructura general del problema sigue siendo no convexa.

En este contexto, se ha estudiado la estructura genérica de los problemas de complementariedad mixta (MPCC) que surgen del enfoque KKT/FJ. Se ha demostrado que, en términos generales, la condición MPCC-LICQ (condición de independencia lineal) se cumple en todos los puntos factibles. Sin embargo, las condiciones de complementariedad estricta (MPCC-SC) y las condiciones de segundo orden (MPCC-SOC) pueden fallar en puntos críticos (estacionarios), incluso en situaciones genéricas \cite{Allende2012SolvingBP}. Esta situación complica aún más la obtención de garantías sobre la solución.

Es importante señalar que existen casos singulares donde los puntos estacionarios pueden ser problemáticos, especialmente cuando un multiplicador (\(\alpha\)) es igual a cero. En tales circunstancias, la condición MPCC-SC puede no cumplirse, lo que podría llevar a que el método KKT no funcione adecuadamente \cite{Allende2012SolvingBP}.

La dificultad para encontrar soluciones globales se ve exacerbada por la tendencia de los métodos de búsqueda local a quedar atrapados en óptimos locales. Por ello, se han desarrollado métodos de búsqueda global que consideran la estructura específica de los problemas de optimización de dos niveles. Estos enfoques suelen incluir fases tanto de búsqueda local como global, utilizando condiciones de optimalidad global para mejorar la efectividad del proceso \cite{DempeyZemkoho2020}.

Además, para abordar la no unicidad de las soluciones y mejorar la estabilidad del sistema, se emplean métodos de regularización. Estos métodos pueden implicar la regularización de la función objetivo del seguidor o del conjunto de respuestas óptimas del mismo \cite{DempeyZemkoho2020}.

En resumen, obtener garantías sobre soluciones en problemas de optimización de dos niveles es un desafío complejo debido a su no convexidad inherente, las dificultades para escapar de óptimos locales y la posible falta de unicidad y estabilidad en las soluciones. Los algoritmos frecuentemente dependen de puntos estacionarios, los cuales no siempre corresponden a las soluciones óptimas deseadas. Por lo tanto, es crucial desarrollar métodos especializados que aborden estos problemas y permitan encontrar soluciones globales o aproximaciones adecuadas \cite{DempeyZemkoho2020}.

% Que se va hacer en la tesis
Debido a la problematica que puede acarrear los puntos estacionationarios con respecto al valor de $\alpha$, los cuales son de interés. 
En esta tesis se propone un generador de problemas que dado un punto estacionario conocido se pueda conocer el comportamiento de este problema en las vecindades del punto los algoritmos para este tipo de problemas de problemas Single-Leader-Single-Follower con enfoque optimista y su comparación con solucionadores del entorno Julia\\.
% Explicación 
% 2do cap
La tesis está compuesta, luego del capítulo de la introducción, por un segundo capítulo donde se precisa la notación a emplear, se define formalmente un problema dos niveles con un líder y un seguidor se explica la teoría matemática para su transformación en un problema MPEC, así como de los algoritmos Julia que usaremos en esta.
% 3ero
En el tercer capítulo explicaremos sobre la implementación algorítmica propuesta anteriormente y su correcta utilización 
% 4to
Y en el cuarto capítulo se analizan los resultados obtenidos por el algoritmo propuesto y su comparación con algoritmos implementados en el entorno Julia.
% Siguiente
Posteriormente se dan las conclusiones y recomendaciones del trabajo realizado.

