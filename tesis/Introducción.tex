\chapter{Introducción}
% Primer parráfo introductorio 
La optimización binivel es una herramienta matemática que permite modelar situaciones complejas donde intervienen dos niveles de decisión jerárquicos. Este enfoque se utiliza en diversas áreas, desde la planificación de redes de distribución hasta la gestión de recursos en entornos industriales. La importancia de estos problemas radica en su capacidad para reflejar decisiones interdependientes, donde un decisor superior (o líder) establece un marco dentro del cual un decisor inferior (o seguidor) toma decisiones que afectan el resultado global.
% Explicación de usos
En \cite{ramos2016} se da un bosquejo de uso de estos 