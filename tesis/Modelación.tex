\chapter{IMPLEMENTACIÓN}
%region Nomenclatura 
En este capítulo se mostrará la forma y los pasos para generar el problema deseado en el contexto de un generador de puntos estacionarios para problemas binivel. Este capítulo aborda el diseño y la implementación del generador, explicando detalladamente las metodologías empleadas, los criterios necesarios para asegurar la validez de los puntos obtenidos y los algoritmos utilizados en cada etapa del proceso.

\section{Nomenclatura a utilizar:}
\begin{table}[H]
    \centering
    \caption{Nomenclatura a utilizar}
    \begin{tabular}{l m{360pt}}
        $ F(x,y) $              & Función del líder.                                                                                                          \\
        $ g_i(x,y) $              & Restricciones de desigualdad del líder   $ i=1\ldots q$.                                                                                                        \\
        $ h_i(x,y) $                 & Restricciones de igualdad del líder   $ i=1\ldots r$.                    \\
        $ f(x,y) $           & Función del seguidor.                                                               \\
        $ v_j(x,y) $              &  Restricciones de desigualdad del seguidor $j=1\ldots s$.   \\
        $ u_j(x,y) $     & restricciones de igualdad del seguidor $j=1\ldots t$. \\
        $ v_{j}^{\star} $    & Restricción de desigualdad del seguidor activa después de modificarse el problema   $j=1\ldots s$.          \\
        $z_{est}$         & Punto que se desea que sea estacionario.\\
        $J_0^g$             & Índices activos del líder. \\
        $J_0^v$                & Índices activos del seguidor. \\
        $ \alpha  $             & Vector de entrada de dimensión igual cantidad variables seguidor.                                                                                                      \\
        $\mu_i $                & Multiplicador asociado al $g_i(x,y)$ en el líder.  \\
        $ \beta_j $               & Multiplicador asociado al $v_{j}^{\star}$ en el líder.          \\
        $ \lambda_j $              & Multiplicador asociado al $v_{j}^{\star}$ en el seguidor.\\
        $\gamma_j$                & Valor asociado a cada $v_{j}^{\star}$ con $\lambda_j=0$.\\
    \end{tabular}

    \end{table}

\newpage
%endregion
\section{Generación del problema}
A continuación se mostrarán los pasos a seguir para generar el problema deseado donde el punto sea estacionario de la clase específica.

\subsection{Requerimientos de entrada}

Para generar un problema es necesario inicialmente introducir un problema de optimización binivel del tipo SLSF, un punto ($z_{est}$), el cual
se desea que sea estacionario y los multiplicadores con signo o valores con ciertas condiciones en dependencia del tipo de estacionariedad requerida.

\begin{samepage}
El problema de entrada tiene que cumplir con ser un programa de optimización binivel como \refeq{eq:DefBinivelOptimista}. 
Debe de conocerse sus índices activos de las restricciones de desigualdad en cada nivel:
\begin{itemize}
    % Leader
    \item Nivel Superior:
            \begin{equation}
             J_0^G=\{i | g_i(x,y)=0\}
            \label{J_0_level_superior} %al conjunto restricciones de desigualdad que sean índices activos del nivel superior
            \end{equation}\\
    donde estos índices activos tienen un $\mu_i$ relacionados.
        
    %Follower
    \item Nivel Inferior:
                Se tiene en cuenta con respecto a los valores de $\lambda_i$ y $v_j(x,y)$.
                
                \begin{itemize}
                    %J_0_v_l_0
                    \item \begin{equation}
                        J_1^v=\{j | v_j(x,y)=0 \land \lambda_j=0 \} %al conjunto restricciones de desigualdad que sean índices activos
                        \label{J_0_lambda_0_level_inferior}
                        \end{equation} 
                    %J_0_v_l_positiva
                    \item \begin{equation}
                        J_2^v=\{j | v_j(x,y)=0 \land \lambda_j>0 \}
                        \label{J_0_lambda_pos_level_inferior}
                    \end{equation}
                    %J_Ne_L0_v
                    \item \begin{equation}
                        J_3^v=\{j | v_j(x,y)< 0 \land \lambda_j=0 \}
                        \label{J_neg_lambda_0_level_inferior}
                    \end{equation}
                \end{itemize}
    Cada índice activo tiene multiplicadores $\beta_j$ y en dependencia del caso $\gamma_j$ asociados.
\end{itemize}
\end{samepage}

% Condiciones
%Hablar sobre conocer vector \alpha
Se debe conocer el valor del $\vec{\alpha}$. 
Además para cada $i \in J_0^G$ (\refeq{J_0_level_superior}) debe de asignarse su $\mu_i \geq 0$ correspondiente,
para el nivel inferior debe tenerse en cuenta si $\alpha=\vec{0}$ en cuyo caso afirmativo $\gamma_j$ es un valor de entrada

\begin{samepage}
% COndiciones \Beta_i \Gamma_i para puntos estacionarios
Para obtener las diferentes clases de puntos estacionarios debe de introducirse $\beta_j$, $\gamma_j$, en caso de ser este último valor de entrada tal que, en los índices activos $J_3^v$ (\refeq{J_0_lambda_0_level_inferior}), como se muestra a continuación: 
\begin{itemize}
    %C estacionario
    %Betai, gammai>0, betai,gammai<0 o 
    %Betai libre gammai=0
    %Gammai libre , betai=0
    \item \textbf{Punto C-Estacionario:}\\
    \begin{equation}
        \beta_j * \gamma_i >=0
        \label{Requisitos puntos C-Estacionario}
    \end{equation}
    %M estacionario
    %Betai, gammai>0,
    %Betai libre gammai=0
    %Gammai libre , betai=0
    \item \textbf{Punto M-Estacionario}\\
    \begin{equation}
        \begin{aligned}
            &\beta_j \land \gamma_j>0\\
            &\beta_j \quad \text{Libre} \quad \gamma_j=0\\
            &\beta_j=0 \quad \gamma_j \quad \text{Libre}\\
        \end{aligned}
        \label{Requisitos puntos M-Estacionarios}
    \end{equation}
    %Fuertemente estacionario
    %Betai, gammai>0,
    \item \textbf{Punto Fuertemente Estacionario}\\
     \begin{equation}
        \beta_j \land \gamma_j >0
    \label{Requisitos puntos Fuertemente Estacionarios}
    \end{equation}    
\end{itemize}
% Especificar que en caso de  que alpha no se el vector nulo se asume que gamma_i es 0
\textbf{En caso que $\vec{\alpha} \neq \vec{0}$ se asume que $\gamma_j=0$ en las condiciones anteriores.}
\end{samepage}
% Explicar que se hace primero el punto factible
% Subseccion donde se explica que el punto se hace factible inicialmente

\subsection{Modificación del Problema Original}
Después de tener los datos de entrada necesarios se procede a la modificación
del problema de entrada para que este sea estacionario del tipo requerido, $z_{est}$.

\begin{samepage}
% Decir que en caso de que \alpha != de nulo entonces se calcula b_j
Primero en caso de que $\alpha \neq 0$ en los $v_j(x,y) \in J_0^v$ se procede a calcular el $\vec{b_j}$ de la siguiente forma:

%Como calcular b_j
\begin{itemize}
    \item \textbf{ $v_j(x,y) \in J_2^v$ \refeq{J_0_lambda_pos_level_inferior}}:
        \begin{equation}
            \vec{b_j}=  \frac{{\alpha} \cdot (-\nabla_{y}{v_j(z_{est})}^T \cdot \alpha)}{\|\mathbf{\alpha} \|_2}
        \end{equation}
    \item \textbf{$v_j(x,y) \in J_1^v \lor J_3^v$ (\refeq{J_0_lambda_0_level_inferior},\refeq{J_neg_lambda_0_level_inferior})}\\
    \begin{equation}
        \vec{b_j}=  \frac{{\alpha} \cdot ((-\nabla_{y}{v_j(z_{est})}^T \cdot \alpha)+\gamma_j)}{\|\mathbf{\alpha} \|_2^2}
    \end{equation}
\end{itemize}


Después de realizado el calculo del $b_j$ se procede a modificar su índice activo correspondiente:

\begin{equation}
	v_{j}^{\star}(z_{est})=v_{j}(z_{est})+ ({\vec{b_j}}^T)\cdot (y_1,y_2,\dots,y_m)
\end{equation}
\end{samepage}

% Explicar que se hace el punto factible en las restricciones
Luego en todas las restricciones del nivel inferior y superior
se evalúa el punto y se verifica si es factible. En caso de no ser factible 
bajo las nuevas restricciones se añaden constantes $c_i$ en el caso de las restricciones del
nivel superior y $c_j$ para las del inferior. Esto conlleva a que sea un punto factible en el conjunto
de restricciones de ambos niveles sin perjudicar las propiedades relacionadas con la convexidad al ser una adicción
de funciones lineales. 

% Explicar que se realiza el KKT del nivel inferior
Al hacer factible se procede a realizar el KKT del nivel inferior con las modificaciones anteriores evaluado en $z_{est}$.

Se plantea las condiciones KKT y se halla $\vec{bf}$ :

\begin{equation}
    \begin{aligned}
        &\nabla_{y}f(x,y)+\sum_{j=0}^{|V \in J_0|}(\lambda_j\nabla_{y}v_{j}^{\star}(x,y))+\vec{bf}=\vec{0}\\
        &\text{KKT del problema del nivel inferior}
    \end{aligned}
    \label{KKT_nivel_inferior}
\end{equation}
\linebreak
%Explicar que como se había hallado ya el KKT del level inf ahora se puede realizar completo
Finalmente podemos construir la reformulación en MPEC como \refeq{eq:KKT_Optimista} calculando el $\vec{BF}$ y 
obteniendo el problema de forma reformulada.
% MPEC all KKT 
\begin{table}[H]
    \begin{equation}
        \label{KKT_del_MPEC}
    \end{equation}
	$\nabla_{xy}F(x,y)+\sum_{i=1}^{|G \in J_o|}(\mu_i\nabla_{xy}g(x,y))+[\nabla_{x,y}\nabla_{y}f(x,y)+\sum_{j=1}^{|V \in J_0|}\lambda_j\nabla_{xy}\nabla_{y}v_{j}^{\star}(x,y)]\alpha+\sum_{j=1}^{|V \in J_0|}(\beta_j\nabla_{xy}v_{j}^{\star}(x,y))+\vec{BF}=\vec{0}$  \label{KKT_del_MPEC}
\caption*{KKT del MPEC}
\end{table}

% Explicación algorítmica y manual de usuario

\section{Implementación Algorítmica y Guía de Usuario}
%region Introducción de la sección
En la implementación computacional del generador propuesto en la sección anteriores
se utilizó el lenguaje de programación \textbf{Julia} \cite{Juliadocs} dado a su versatilidad
en expresiones y funciones matemáticas implementadas en su paquete base y sus bibliotecas externas como 
\textbf{BilevelJuMP} \cite{BilevelJump} que permite resolver problemas binivel SLSF lineales y cuadráticos
sin tener que transformar el problema original y \textbf{JuMP} \cite{JuMPPaper} el cual, es una interfaz robusta para la optimización general,
todo ello con unas excelentes prestaciones de cómputo. Por ello se brindará una guía de usuario para el uso del generador.

%\subsection{Guía de Usuario}
% Después compilar a ProblemGenerator
La implementación algorítmica se ha llevado a cabo mediante la creación de una biblioteca de Julia
llamada \textbf{ProblemGenerator}, con una sintaxis \textit{JuMP-like} intuitiva.
Primero debe tenerse un problema de optimización Binivel SLSF planteado como el \refeq{eq:DefBinivelOptimista}, el punto que se desea a ser estacionario ($z_{est}$), los índices activos descritos anteriormente 
según el caso así como sus multiplicadores correspondientes al tipo de estacionariedad requerida y el $\vec{\alpha}$.
%endregion

\subsection{Ejemplo a utilizar}
Para una mayor facilidad de compresión del uso de la biblioteca se utiliza este problema ejemplo:
\begin{equation}
    \begin{aligned}
        & \underset{x_1, x_2}{\text{mín}} 
        && x_1^2 y_1^2 y_2 + x_2 \\
        & \text{s.t.} 
        && x_1 + y_2 - y_1 > 9, \\
        & 
        && \underset{y_1, y_2}{\text{mín}} 
        \quad x_2^2 y_1^2 y_2 + x_1 \\
        & 
        && \text{s.t.} 
        \quad x_1^2 y_1^2 + x_2 \leq 0.
    \end{aligned}
    \label{ProblemaEjemplo}
\end{equation}
%Dependencias para la biblioteca
\subsection{Dependencias}
Se debe tener instalado en el entorno de desarrollo de Julia los siguientes módulos:
%Se debe tener en cuenta que este paquete instalará las siguientes dependencias
\begin{itemize}
    \item Symbolics 
    \item LinearAlgebra
\end{itemize}

% Importar la Biblioteca
\subsection{Importación de la biblioteca}
Para la utilización de la biblioteca se procese a su importación de la siguiente forma:
\begin{lstlisting}[caption=Importar el Módulo]
    using ProblemGenerator
\end{lstlisting}

% Crear modelo base
\subsection{Crear el modelo base}
Para comenzar debe llamarse a la función \textbf{GeneratorModel} para crear el modelo base
inicial.

\subsubsection{Para $\vec{\alpha}=\vec{0}$ }

%Crear para alpha nulo
Para crear un modelo donde $\vec{\alpha}=\vec{0}$:
   
        \begin{lstlisting}[caption={Crear el modelo para $\vec{\alpha}=\vec{0}$}]
            # Crear el modelo base del generador alpha=0
            model=GeneratorModel()
        \end{lstlisting}
        
    
  
%Crear para alpha no nulo
\begin{samepage}
\subsubsection{Para $\vec{\alpha}= \neq vec{0}$ }

Para crear un modelo donde $\vec{\alpha}\neq \vec{0}$,
se tiene que pasar como parámetro el valor del $\alpha$
el cual será un vector de \textit{Number}:

\begin{lstlisting}[caption={Crear el modelo para $\vec{\alpha } \neq \vec{0}$}]
    # Crear el modelo base del generador alpha!=0
    alpha_vec=Vector::{Number}
    model=GeneratorModel(alpha_vec)
\end{lstlisting}
\end{samepage}



% Explicar las funciones Upper y Lower
\subsection{Función para diferenciar los niveles}
En caso que se requiera como entrada un \textbf{Problem} se debe declarar
de que nivel se trata mediante evaluar el model en las funciones \textbf{Upper} para el nivel 
superior y \textbf{Lower} para el inferior.

\subsubsection{Para referirse al nivel superior}
\begin{lstlisting}[caption={Referirse al nivel superior}]
    # Referirse nivel superior
    Upper(model)
\end{lstlisting}
\subsubsection{Para referirse al nivel inferior}
\begin{lstlisting}[caption={Referirse al nivel inferior}]
    # Referirse nivel inferior
    Lower(model)
\end{lstlisting}


\subsection{Declarar Variables}
Las variables deben de 
declararse bajo la siguiente macro \textbf{@myvariables}, la cual 
recibe una función \textbf{Upper} o \textbf{Lower} que recibe el modelo
para las variables del nivel superior y las del nivel inferior respectivamente.

\subsubsection{Introduccion de  las variables del nivel superior }
%Ejemplo de introducir al level superior

\begin{lstlisting}[caption={Introducir las variables del nivel superior}]
    # Se declara en el nivel superior las variables x_1, x_2
    @myvariables Upper(model) x_1, x_2
\end{lstlisting}

\subsubsection{Introducción de las variables del nivel inferior}
%Ejemplo para introducir las variables del nivel inferior

\begin{lstlisting}[caption={Introducir las variables del nivel inferior}]
   # Se declara en el nivel inferior las variables y_1, y_2
   @myvariables Lower(model) y_1,y_2
\end{lstlisting}


%Declarar las funciones objetivo
\subsection{Declarar Funciones Objetivo}

Para declarar las funciones objetivos debe asumirse que en ambos casos es un 
problema de minimización. Se utilizará la función \textbf{SetObjectiveFunction}
que recibe un \textbf{Problem} y una expresión de \textbf{Num}.

% Declaración de Función Objetivo de nivel Superior
\subsubsection{ Declaración de una función objetivo del nivel superior}
Con: $$\min (x_1^2*y_1^2*y_2) + x_2$$
\begin{lstlisting}[caption={ Declarar una función objetivo del nivel superior}]
    # Declarar la funcion objetivo del nivel superior
    # Min de ((x_1^2)*(y_1^2)*(y_2))+x_2
    SetObjectiveFunction(Upper(model),((x_1^2)*(y_1^2)*(y_2))+x_2)
\end{lstlisting}

\subsubsection{Declaración de una función objetivo del nivel inferior}
%Declaración de Función objetivo Nivel inferior

Con: $$\min (x_2^2*y_1^2*y_2)+x_1$$
\begin{lstlisting}[caption={Declarar una función objetivo del nivel inferior.}]
    # Declarar la funcion objetivo del nivel inferior
    # Min de ((x_2^2)*(y_1^2)*(y_2))+x_1
    SetObjectiveFunction(Lower(model),((x_2^2)*(y_1^2)*(y_2))+x_1)
\end{lstlisting}

% Definición de tipo de índices activos.
\subsection{Definición del conjunto de Índices activos}
\begin{samepage}
Antes de ilustrar como introducir las restricciones se va a explicar 
que los tipos de índices activos.

\begin{itemize}
    \item \textbf{Ambos Niveles}:
        \begin{itemize}
            \item \textbf{Normal} si no es un índice activo.
        \end{itemize}
    \item \textbf{Nivel Superior}:
     \begin{itemize}
        \item \textbf{J\_0\_g} Si es un índice como en \refeq{J_0_level_superior}.
     \end{itemize}
    \item \textbf{Nivel Inferior}:
    \begin{itemize}
        \item \textbf{J\_0\_LP\_v} Si es un índice como en \refeq{J_0_lambda_pos_level_inferior}.
        \item  \textbf{J\_0\_L0\_v} Si es un índice como en \refeq{J_0_lambda_0_level_inferior}.
        \item  \textbf{J\_Ne\_L0\_v} Si es un índice como en \refeq{J_neg_lambda_0_level_inferior}.
    \end{itemize}
\end{itemize}
\end{samepage}

%Ejemplo de como se debe de expresar en Julia
Los \textbf{RestrictionSetType} deben expresar en código de esta forma:
\begin{lstlisting}[caption={Definir el conjunto de índice activo}]
    Normal 
    J_0_g 
    J_0_LP_v
    J_0_L0_v
    J_Ne_L0_v
\end{lstlisting}

Los \textbf{RestrictionSetType} son un enum de Julia.

% Declaración de restricciones
\subsection{Declarar Restricciones}

Para declarar las restricciones se brindan dos funciones:
\begin{itemize}
    \item \textbf{SetLeaderRestriction:} Para el nivel superior.
    \item \textbf{SetFollowerRestriction:} Para el nivel inferior.
\end{itemize}
\begin{itemize}
    % Declarar sobre el nivel superior
    \item \textbf{Nivel Superior:}\\

% Declaración para el nivel Superior
Para declarar las restricciones del nivel superior debe por cada restricción
llamarse a la función \textbf{SetLeaderRestriction} con:
\begin{itemize}
    \item El modelo (\textbf{model})
    \item La expresión de la restricción, del tipo \textbf{Num}
    \item El tipo de restricción, del tipo \textbf{RestrictionSetType}
    \item El valor de $\mu_i$ correspondiente, del tipo \textbf{Number} 
\end{itemize}

Ejemplo para introducir la restricción:
\begin{align*}
    &x_1+y_2-y_1 > -9 \\
    &\text{Índice activo del tipo J\_0\_g \refeq{J_0_level_superior}}\\
    & \mu_i=0.3
\end{align*}

\begin{lstlisting}[caption={Introducir restricción del nivel superior}]
    # Ejemplo de restriccion del nivel superior
    SetLeaderRestriction(model,x_1+y_2-y_1>9,J_0_g,0.3)
\end{lstlisting}

\begin{samepage}
    % Declarar sobre el nivel inferior
\item \textbf{Nivel Inferior:}\\
Para declarar las restricciones del nivel inferior debe por cada restricción
llamarse a la función \textbf{SetFollowerRestriction} con:
\begin{itemize}
   \item El modelo (\textbf{model})
   \item La expresión de la restricción, del tipo \textbf{Num}
   \item El tipo de restricción, del tipo \textbf{RestrictionSetType}
   \item El valor de $\beta_j$ correspondiente, del tipo \textbf{Number}
   \item El valor de $\lambda_j$ correspondiente, del tipo \textbf{Number}
   \item El valor de $\gamma_j$ correspondiente en caso de ser valor de entrada, del tipo \textbf{Number}
\end{itemize}

\end{samepage}
Ejemplo para introducir la restricción:

\begin{itemize}
    \begin{samepage}
    \item \textbf{$ \lambda_j \neq 0$:}\\
    \begin{align*}
        &((x_1^2)*(y_1^2))+x_2=0\\
        &\text{Índice activo del tipo J\_Ne\_L0\_v} \refeq{J_neg_lambda_0_level_inferior}\\
        & \beta_j=0.1\\
        & \lambda_j=0.5 \\
    \end{align*}
    \begin{lstlisting}
        # Para caso lamda_j=0
        SetFollowerRestriction(model,((x_1^2)*(y_1^2))+x_2<=0,J_Ne_L0_v,0.1,0.5,0)
    \end{lstlisting}
    \end{samepage}

    \begin{samepage}
   
    \item \text{$\lambda_j= 0$:}\\
    
    Análogo al caso anterior pero con $\lambda_j=0$ por lo que $\gamma_j$ valor de entrada
    \begin{align*}
        &  \beta_j=0.1\\
        &\lambda_j=0\\
        &\gamma_j=0.4
    \end{align*}
    \begin{lstlisting}
        # Para caso lamda_j!=0
        SetFollowerRestriction(model,((x_1^2)*(y_1^2))+x_2<=0,J_Ne_L0_v,0.1,0,0.4)
    \end{lstlisting}
\end{samepage}
\end{itemize}


% Finalizar el itemize que tiene de explicación como introducir las restricciones de ambos niveles
\end{itemize}


\subsection{Introducir el Punto}
Ahora debe de introducirse el valor del punto que debe ser estacionario de la clase seleccionada.
Debe de tenerse en cuenta que para todas las variables declaradas en ambos niveles debe de definirse el valor de la componente.

\begin{lstlisting}[caption={Introducir el punto $(1,1,1,1)$}]
    # Introducir el punto (1,1,1,1)
    SetPoint(model,Dict(x_1=>1,x_2=>1,y_1=>1,y_2=>1))
\end{lstlisting}

\subsection{Generar el Problema}
Finalmente al tener todos los datos previos introducidos se llama al \textbf{CreateProblem} pasándole como parámetro el \textit{model} 
y generará dicho problema imprimiendo en consola este.

\begin{lstlisting}[caption={Generar el problema}]
    # Llamar para generar el problema
    problem=CreateProblem(model)
\end{lstlisting}
\begin{samepage}
    
\subsection{Ejemplo completo}
Se muestra el ejemplo completo para el problema antes propuesto:
\begin{lstlisting}[caption={Script}]
    # Importar dependencias necesarias
    using ProblemGenerator
    # Crear el modelo base del generador alpha=0
    model=GeneratorModel()
    #Declaracion de variables
    # Se declara en el nivel superior las variables x_1, x_2
    @myvariables Upper(model) x_1, x_2
    # Se declara en el nivel inferior las variables y_1, y_2
    @myvariables Lower(model) y_1,y_2
    # Declarar la funcion objetivo del nivel superior
    # Min de ((x_1^2)*(y_1^2)*(y_2))+x_2
    SetObjectiveFunction(Upper(model),((x_1^2)*(y_1^2)*(y_2))+x_2)
    # Ejemplo de restriccion del nivel superior
    SetLeaderRestriction(model,x_1+y_2-y_1>9,J_0_g,0.3)
    # Declarar la funcion objetivo del nivel inferior
    # Min de ((x_2^2)*(y_1^2)*(y_2))+x_1
    SetObjectiveFunction(Lower(model),((x_2^2)*(y_1^2)*(y_2))+x_1)
    # Para caso lamda_j!=0
    SetFollowerRestriction(model,((x_1^2)*(y_1^2))+x_2==0,J_Ne_L0_v,0.1,0,0.4)
    # Introducir el punto (1,1,1,1)
    SetPoint(model,Dict(x_1=>1,x_2=>1,y_1=>1,y_2=>1))
    # Llamar para generar el problema
    problem=CreateProblem(model)
\end{lstlisting}

\end{samepage}
\subsection{Documentación Oficial}
Para mayor información visitar la \href{https://github.com/FVSB/Tesis}{documentación oficial del generador}. 

Se pone antes de entrega rel host donde esta la pagina de MKDOCS
%TODO: Referenciar al link de usuario