\chapter{MODELACIÓN MATEMÁTICA E IMPLEMENTACIÓN}

Dado que por estudios previos (Nombrar )\\

Se tomará transformará el MPEC de la siguiente forma


\begin{table}[H]
    \begin{tabular}{l m{360pt}}
        $ F(x,y) $              & Función del líder                                                                                                          \\
        $ G(x,y) $              & Restricciones de desigualdad del líder                                                                                                            \\
        $ H(x,y) $                 & Restricciones de igualdad del lider                       \\
        $ f(x,y) $           & Funcion del follower                                                               \\
        $ V(x,y) $              &  Restricciones de desigualdad del follower  \\
        $ U(x,y) $     & restricciones de igualdad del follower\\
        $ V_{j}^{\star} $    & Restriccion de disigualdad del follower activa despues de modificarse el problema            \\
        $ \alpha  $             & Vector de entrada del problema puede ser el vector nulo  dimension igual que Cant V activas                                                                                                        \\
        $ \beta $               & Multiplicador asociado al $V_{j}^{\star}$ en el nivel superior          \\
        $ \lambda $              & Multiplicador asociado al $V_{j}^{\star}$ en el nivel inferior\\
        $\gamma$                & Valor asociado a cada $V_{j}^{\star}$ con $\alpha=\vec{0}$\\
    \end{tabular}

    \end{table}
%Notas de J_0^v
        %\begin{table}
		%		Nota:
        %	\begin{itemize}
        %	    \item Sea $ J_0^v=\{j | V_i(x,y)=0\}$ al conjunto restricciones de desigualdad que sean índices activos
		%	
        %	\end{itemize}
		%\end{table}
Se propone un generador de problemas prueba que dado un problema de optimización binivel y un punto 
se pueda conocer la factibilidad de los algoritmos tradicionales en dicho puntos (Explicar lo de los puntos estacionarios y ptos dengenerados )
Con este se puede conocer para los casos en que $\alpha=0$ y $\alpha>0$ y ver el comportamiento en estos.


Para utilizar el m'etodo propuesto debe darse como valores de entrada 


Se debe determinar cuales indices activos sus $\lambda = 0$ y cuales son $\lambda>0$ con el fin de poder calcular el valor de $b_j$ el se hará el siguente procedimiento

\begin{table}[H]
	$\lambda =0$:\\
	$(\nabla_{y}V_j+b_j)\alpha=0 \quad \forall V_j \in J_0$ \\
	$\lambda>0$:\\
	$[(\nabla_{y}V_j+b_j)\alpha]-\gamma_j=0 \quad \forall V_j \in J_0$


\end{table}

Despues de realizado el calculo del $b_j$ se procede a modificar su indice activo correspondiente:

\begin{table}[H]
	$V_{j}^{\star}=V_{j}+(b_j\alpha)$
\end{table}

Después se realiza las adicciones necesarias de una constante a cada restriccion con el fin de que sea factible el punto seleccionado
esta suma de constante no modifica las propiedades de convexidad del problema dado que una suma de funciones lineales no afecta.\\

% Decir que se realiza el KKT del nivel inferior

Despues se procede a realizar el KKT del nivel inferior con las modificiaciones anteriores 
Se plantea las condiciones KKT de las funciones evaluadas en el punto dado:
\begin{table}[H]
	$\nabla_{y}f(x,y)+\sum_{j=0}^{|V \in J_0|}(\lambda_j\nabla_{y}V_j(x,y))+\vec{bf}=\vec{0}$
\caption*{KKT del problema del nivel inferior}
\end{table}


Una vez hallado el KKT del nivel inferior se procede a hallar el KKT del MPEC resultante de este KKT el cual es:

% MPEC all KKT 
\begin{table}[H]
	$\nabla_{xy}F(x,y)+\sum_{i=1}^{|G \in J_o|}(\mu_i\nabla_{xy}g(x,y))+[\nabla_{x,y}\nabla_{y}f(x,y)+\sum_{j=1}^{|V \in J_0|}\lambda_j\nabla_{xy}\nabla_{y}V_j(x,y)]\alpha+\sum_{j=1}^{|V \in J_0|}(\beta_j\nabla_{xy}V_j(x,y))+\vec{BF}=\vec{0}$
\caption*{KKT del MPEC}
\end{table}

