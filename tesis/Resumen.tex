\chapter*{Resumen}

El problema de optimización binivel se define como minimizar una función sobre un conjunto determinado por los puntos óptimos de un modelo de programación matemática. La optimización en el nivel inferior depende de las decisiones tomadas en el nivel superior, creando así una relación de interdependencia entre ambos niveles.

Para abordar este problema, se considera la creación de un problema relacionado en el cual el nivel inferior se sustituye por las condiciones necesarias de optimalidad, siendo este un problema con restricciones de complementariedad.

En este trabajo se propone una forma de generar problemas de dos niveles cuyo problema con restricciones de complementariedad (MPEC) tiene un punto estacionario perteneciente a una de las siguientes 
clases: fuertemente estacionario, M-estacionario o C-estacionario, dependiendo de los multiplicadores.


\textbf{Palabras clave:} Optimización binivel, MPECs, Condiciones necesarias de optimalidad, Punto estacionario.

\chapter*{Abstract}
The bilevel optimization problem is defined as minimizing a function on a set determined by the optimal points of a mathematical programming model. The optimization at the lower level depends on the decisions made at the upper level, thus creating an interdependent relationship between both levels.

To address this problem, the lower-level problem is replaced by the necessary optimality conditions and the resulting (relaxed) optimization problem with complementarity constraints is solved.

This work proposes a way to generate two-level problems whose relaxed problem has a stationary point belonging to one of the following classes: strongly stationary, M-stationary, or C-stationary, depending on the multipliers.

\textbf{Keywords:} Bi-level optimization, MPECs, Necessary optimality conditions, Stationary point.
