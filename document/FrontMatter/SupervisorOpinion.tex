\begin{opinion}
    La optimización binivel es un problema inherentemente complejo debido a su estructura jerárquica: las decisiones tomadas en el nivel superior condicionan las opciones disponibles en el nivel inferior, y la solución óptima del nivel inferior, a su vez, influye en la función objetivo del nivel superior. Este ciclo de retroalimentación hace que encontrar la solución óptima global sea particularmente desafiante.

Una estrategia común para abordar estos problemas es su reformulación como un MPEC (\textit{Mathematical Program with Equilibrium Constraints} o Problema de Programación Matemática con Restricciones de Equilibrio). Sin embargo, las restricciones de complementariedad propias de los MPEC introducen dificultades adicionales en su resolución.

La contribución principal de esta tesis es la generación de problemas cuya transformación en un MPEC posee un punto estacionario con propiedades teóricamente y algorítmicamente relevantes. Contar con una metodología para generar problemas con puntos estacionarios específicos puede resultar muy valioso para evaluar el rendimiento de distintos algoritmos de solución de MPEC, así como para comprender mejor la relación entre la estructura del problema y las características de sus soluciones. Además, este enfoque se desarrolla preservando propiedades clave como la linealidad y la convexidad de las funciones involucradas, gracias a la incorporación de funciones lineales.

Para un estudiante de Computación, desarrollar esta tesis no ha sido una tarea sencilla. Francisco ha debido sumergirse en el complejo campo de la optimización binivel, programar el generador de problemas, familiarizarse con Julia y comprender los algoritmos de solución aplicables.

A lo largo de este proceso, Francisco ha demostrado ser un estudiante trabajador, meticuloso e interesado, enfrentando con éxito los retos que se le han presentado. Su capacidad para abordar de manera independiente los desafíos de esta investigación es destacable. Confío en que la experiencia adquirida le será de gran utilidad en su futuro profesional.
\end{opinion}