\chapter{Conclusiones}

%Explicar que se hizo
En esta tesis se desarrolló un algoritmo para la generación de problemas de optimización binivel con características específicas de estacionariedad en puntos determinados. El mismo tiene la capacidad de modificar problemas originales para garantizar la factibilidad y estacionariedad en puntos dados, aprovechando las capacidades del lenguaje de programación Julia, que destaca por su alto rendimiento computacional y sintaxis adaptada a problemas de optimización.

% Explicar como fue la experimentación
La experimentación se llevó a cabo sobre tres categorías fundamentales de problemas: lineales, cuadráticos y no convexos. El proceso experimental comenzó con la obtención de puntos mínimos utilizando bibliotecas establecidas como BilevelJuMP y JuMP. Posteriormente, se generaron problemas estacionarios mediante la adición de componentes aleatorias a estos puntos para cada clase de problema definida.

% Que se comparó
Los problemas modificados fueron sometidos nuevamente a algoritmos tradicionales implementados en Julia para evaluar su efectividad frente a los puntos estacionarios generados. Se realizó un análisis comparativo destacando los casos más relevantes de cada categoría de puntos estacionarios, considerando la evaluación de la función objetivo del nivel superior en el punto donde se garantizó la estacionariedad, contrastándola con los resultados obtenidos por las bibliotecas convencionales.

% Recomendaciones
Como resultado de la investigación realizada, se han identificado varias líneas de trabajo futuro que permitirían expandir y mejorar los resultados obtenidos. En primer lugar, se recomienda ampliar el alcance de la experimentación numérica para incluir una mayor diversidad de problemas de optimización binivel. Esta expansión permitiría validar la robustez y versatilidad del algoritmo propuesto en diferentes contextos y escenarios de aplicación.

En segunda instancia, se sugiere profundizar en la investigación sobre la implementación del algoritmo desarrollado como criterio de parada en nuevos métodos de optimización binivel. Esta línea de investigación podría contribuir significativamente al desarrollo de algoritmos más eficientes y confiables, mejorando la capacidad de detectar y verificar puntos estacionarios durante el proceso de optimización.

Finalmente, se propone el desarrollo de una interfaz gráfica más intuitiva y funcional que facilite la generación automática de puntos según el tipo de estacionariedad requerida. Esta mejora en la usabilidad del software permitiría que usuarios con diferentes niveles de experiencia puedan aprovechar las capacidades del algoritmo de manera más efectiva, ampliando así su aplicabilidad práctica en diversos campos de estudio.