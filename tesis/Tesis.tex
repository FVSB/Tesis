\documentclass[spanish, 12pt]{thesis}
\usepackage[T1]{fontenc}
\usepackage[utf8]{inputenc}
\usepackage{lmodern}
\usepackage[a4paper]{geometry}
\usepackage[spanish]{babel}
\usepackage{hyperref}
\usepackage{csquotes}

% Bibliorafía
\usepackage[style=apa,sorting=nyt,backend=biber]{biblatex}
\DeclareLanguageMapping{spanish}{spanish-apa}
\bibliography{references}
%

%\usepackage[]{parskip}
\usepackage{multicol}
\usepackage{multirow}
\usepackage[x11names,table]{xcolor} 
\usepackage{floatrow}
\usepackage{graphicx}
\usepackage{amsmath, latexsym, amsfonts, amssymb} 
\usepackage{booktabs}
\usepackage{url}
%\usepackage{float}
\usepackage[shortlabels]{enumitem}
\usepackage{tikz}
\usepackage{floatrow}
\usepackage[bf,labelsep=period]{caption}
\captionsetup{justification = centering}
\renewcommand{\baselinestretch}{1.5} % interlineado
\usepackage[bottom]{footmisc}

\usepackage{mathtools}
\usepackage{optidef}

% Se añadió ahora para las referencias
\usepackage{cleveref}
% Comentarios
%\usepackage{comment}
%\usepackage{todonotes}
%\usepackage{easyReview}
%

% Python
\usepackage{listings}
%\usepackage{listingsutf8}
\usepackage{color}


% Redefinir el nombre del listing
\renewcommand{\lstlistingname}{Ejemplo} % <--- AQUÍ EL CAMBIO

\definecolor{mygreen}{rgb}{0,0.6,0}
\definecolor{mygray}{rgb}{0.5,0.5,0.5}
\definecolor{mymauve}{rgb}{0.58,0,0.82}
\definecolor{myred}{rgb}{0.58,0,0}


\lstset{ 
	backgroundcolor=\color{white},
	basicstyle=\footnotesize,
	breakatwhitespace=false,  
	breaklines=true,         
	captionpos=b,            
	commentstyle=\color{mygray},   
	deletekeywords={...},   
	escapeinside={\%*}{*)},         
	extendedchars=true,             
	firstnumber=1,               
	frame=,	     
%	identifierstyle=\color{blue},             
	keepspaces=true,                
	keywordstyle=\bfseries\color{mygreen},      
	language=Python,                
	morekeywords={*, ..., Var, AbstractModel, Param, Set, Obective, Constraint, Complementarity},            
	numbers=none,   % Posición de los números de línea (none, left, right).                 
	numbersep=5pt,                   
	numberstyle=\tiny\color{mygray},
	rulecolor=\color{blue},        
	showspaces=false,                
	showstringspaces=false,          
	showtabs=false,                
	stepnumber=1,                    
	stringstyle=\color{myred},     
	tabsize=2,	                  
	title=\lstname                  
}
%

\newcommand{\R}{\mathbb{R}} 
\newcommand{\N}{\mathbb{N}} 
\newcommand{\Z}{\mathbb{Z}} 
\newcommand{\Lagr}{\mathcal{L}}
% Agregar argmin
% En el preámbulo del documento
\usepackage{amsmath}
\DeclareMathOperator*{\argmin}{argmin} % Define el operador argmin

% Dar formato codigo de julia
\usepackage{listings}
\usepackage{xcolor}

% Configuración para código Julia
\usepackage{listings}
\usepackage{xcolor}
\usepackage{cmap} % Mejora la capacidad de copiar/pegar
\usepackage[T1]{fontenc}
% Definición manual del lenguaje Julia
\lstdefinelanguage{Julia}%
{
  keywords={abstract,break,case,catch,const,continue,do,else,elseif,end,%
    export,false,for,function,if,immutable,import,importall,in,let,macro,%
    module,mutable,null,primitive,quote,return,struct,switch,true,try,type,%
    using,while},%
  sensitive=true,%
  alsoother={$},%
  morecomment=[l]\#,%
  morecomment=[n]{\#=}{=\#},%
  morestring=[s]{"}{"},%
  morestring=[m]{'}{'},%
}[keywords,comments,strings]

% Configuración del estilo
\lstdefinestyle{julia-style}{
    language=Julia,
    basicstyle=\ttfamily\small,
    breaklines=true,
    showstringspaces=false,
    commentstyle=\color{green!60!black},
    keywordstyle=\color{blue},
    stringstyle=\color{red},
    frame=single,
    numbers=left,
    numberstyle=\tiny\color{gray},
    backgroundcolor=\color{gray!10},
    captionpos=b,
    keepspaces=true
}

\lstset{style=julia-style}


% Para que se ajusten automáticamente las tablas 
\usepackage[utf8]{inputenc}
\usepackage{geometry}
\usepackage{adjustbox}

% Hoja a4
%\geometry{a4paper, margin=1in}
% Hoja Carta
%\geometry{letterpaper, margin=1in}
\newenvironment{resultstable}[1]{
    \begin{table}[h!]
        \centering
        \caption{#1}
        \begin{adjustbox}{max width=\textwidth}
        \begin{tabular}{| l | l | l | l | l | l | l | }
            \hline
            \textbf{Tipo de punto} & \textbf{Nombre del problema} & \textbf{Punto estacionario} & \textbf{Valor objetivo del punto estacionario} & \textbf{Punto óptimo} & \textbf{Valor del punto óptimo} & \textbf{Método seleccionado}\\
            \hline
}{
        \end{tabular}
        \end{adjustbox}
    \end{table}
}

\newcommand{\resultrow}[7]{
    #1 & #2 & #3 & #4 & #5 & #6 & #7 \\
    \hline
}

% Pone automático los números de las ecuaciones
\numberwithin{equation}{chapter}
% Definir un nuevo entorno para teoremas 
\newtheorem{theorem}{\textbf{Teorema}}
% Definir un nuevo entorno para las definiciones 
\newtheorem{definition}{\textbf{Definición}}

\usepackage{amsmath}
\usepackage{amssymb}
\usepackage{xcolor}
\usepackage{framed}

% Definir el entorno Example
\newenvironment{Example}[1][]{%
    \def\examplecaption{#1}%
    \begin{leftbar}%
    \par\noindent%
}{%
    \end{leftbar}%
    \ifx\examplecaption\empty\else%
    \vspace{5pt}%
    \noindent%
    {\footnotesize\centering\textbf{Example:} \examplecaption\par}%
    \fi%
}


\begin{document}
	\frontmatter
	\begin{titlepage}
	
	\title{\vspace*{-70pt}
		\begin{figure}
			\includegraphics[scale=0.2]{img/logo.png}
		\end{figure} 
		\vspace*{10pt}Facultad de Matemática y Computación \\
		Universidad de La Habana \\
		\vspace*{20pt}	
		Tesis de diploma de la especialidad \\ 
		Ciencia de la Computación \\
		
		\vspace*{20pt}
		Un generador de problemas prueba para evaluar la calidad de la solución de los algoritmos de problemas de optimización de dos niveles 
		\author{{\large Autor: Francisco Vicente Suárez Bellón \vspace*{15pt}} \\
			{\large Tutora: Dr. C. Gemayqzel Bouza Allende} \\ \hspace{10pt}  }
		
		\date{La Habana, \today}
	}
	
\end{titlepage}

	\maketitle
	
	\chapter*{Resumen}

El problema de optimización binivel se define como minimizar una función sobre un conjunto determinado por los puntos óptimos de un modelo de programación matemática. La optimización en el nivel inferior depende de las decisiones tomadas en el nivel superior, creando así una relación de interdependencia entre ambos niveles.

Para abordar este problema, se considera la creación de un problema relacionado en la cual el nivel inferior se sustituye por las condiciones necesarias de optimalidad.

En este trabajo se propone una forma de generar problemas de dos niveles cuyo problema relajado tiene un punto estacionario perteneciente a una de las siguientes clases: strongly stationary, M-stationary o C-stationary, dependiendo de los multiplicadores.


\textbf{Palabras clave:} Optimización binivel, MPECs, Condiciones necesarias de optimalidad, Punto estacionario.

\chapter*{Abstract}
The bilevel optimization problem is defined as minimizing a function over a set determined by the optimal points of a mathematical programming model. The optimization at the lower level depends on the decisions made at the upper level, thus creating an interdependent relationship between both levels.

To address this problem, a relaxation is considered in which the lower-level problem is replaced by the necessary optimality conditions.

This work proposes a way to generate two-level problems whose relaxed problem has a stationary point belonging to one of the following classes: strongly stationary, M-stationary, or C-stationary, depending on the multipliers.

\textbf{Keywords:} Bi-level optimization, MPECs, Necessary optimality conditions, Stationary point.

	
	\include{Agradecimientos.tex}
	
	\tableofcontents
		
	\mainmatter
	\chapter{Introducción}
% Obtener una garantía de la solución es muy complejo (generalmente es estacionario y con puntos estacionarios).
La optimización de dos niveles, un área fundamental en la investigación operativa y la teoría de juegos, presenta desafíos significativos debido a su complejidad inherente. Este tipo de problemas se caracteriza por la interacción entre un líder y un seguidor, donde las decisiones del líder afectan las respuestas del seguidor. Uno de los aspectos más críticos de esta problemática es garantizar la existencia de soluciones óptimas, lo cual se ve complicado por la naturaleza no convexa del problema, incluso cuando las funciones y los conjuntos factibles son convexos. A menudo, los algoritmos utilizados en este contexto solo logran identificar puntos estacionarios o críticos, que no necesariamente representan soluciones locales o globales \cite{DempeyZemkoho2020}.

% Definición de problema binivel
%\begin{equation}
%    \begin{aligned}
%    \text{minimizar} & \quad F(x, y) \\
%    \text{sujeto a} & \quad G(x, y) \geq 0 \quad (\text{restricciones de desigualdad}) \\
%    & \quad H(x, y) = 0 \quad (\text{restricciones de igualdad}) \\
%    & \quad y \in S(x) = \arg \min_{y} \{ f(x, y) \mid V(x, y) \geq 0, U(x, y) = 0 \}.
%    \end{aligned}
% \end{equation}

\begin{table}[H]

    \[\begin{array}{l}
        \underset{\substack{x}}{\min} \quad F(x, y)\\
        s.a \left\{ \begin{array}{l}
            x \in T \\
             y \in S(x) = \arg  \underset{\substack{y}}{\min} \{ f(x, y) \quad s.a \quad y \in  H \}\\
            x,y \in M \\
        \end{array}\right.
    \end{array}\]\\


    \caption*{Problema de Optimización Binivel}
    \end{table}

% Explicar dimensiones 
Donde $x \in R^{n}$, $y \in R^{m}$, $F, f : \mathbb{R}^{n} \times \mathbb{R}^{m} \to \mathbb{R}$,  $T , f :\mathbb{R}^{n} \to \mathbb{R}$, 
$f, f : \mathbb{R}^{n} \times \mathbb{R}^{m} \to \mathbb{R} $, $H , f: \mathbb{R}^{m} \to \mathbb{R}$, $M , f : f : \mathbb{R}^{n} \times \mathbb{R}^{m} \to \mathbb{R}$


% Aplicaciones 
Esta es una herramienta fundamental para modelar y analizar mercados eléctricos complejos, ofreciendo una perspectiva única sobre las interacciones estratégicas entre diversos agentes económicos.
% Mercado Eléctrico
% Nash equilibrium in a pay-as-bid electricity market Part 2 - best response of a producer Didier
En el trabajo de \cite{Aussel2017NashEI}, se profundiza en el análisis de mercados de electricidad de pago por oferta, explorando cómo un productor puede ajustar su estrategia considerando las acciones de sus competidores. El estudio destaca la aplicación de conceptos de \textbf{equilibrio de Nash} y técnicas de mejor respuesta, proporcionando una metodología sofisticada para optimizar la participación de un productor en el mercado.
% Deregulated electricity markets with thermal losses and production bounds: models and optimality conditions
Continuando con esta línea de investigación, \cite{Aussel2016DeregulatedEM} desarrollaron un modelo innovador que aborda los mercados de electricidad desregulados. Su enfoque se distingue por incorporar restricciones de producción y pérdidas térmicas, lo que permite una modelización más precisa y realista. Mediante el uso de modelos binivel, los investigadores pueden explorar escenarios más complejos y representativos del funcionamiento real de los mercados energéticos.
% Electricity spot market with transmission losses
Además en otros trabajos se tienen en cuenta en los modelos las pérdidas de transmisión como en  \cite{Aussel2013ElectricitySM}, donde esta contribución mejora significativamente la representación del sistema eléctrico, permitiendo un análisis más profundo del equilibrio estratégico mediante técnicas de \textbf{optimización}. Al considerar las pérdidas de transmisión, el modelo captura aspectos fundamentales de la distribución y comercialización de energía que anteriormente pasaban desapercibidos.

% Machine Learning
También tiene aplicaciones fundamentales en la selección de hiperparámetros en aprendizaje automático, como lo demuestra el trabajo de \cite{DempeyZemkoho2020ML}. El capítulo 6 del libro aborda la optimización de hiperparámetros en problemas de clasificación y regresión, presentando algoritmos innovadores para manejar funciones objetivo no suaves y no convexas. La razón del uso de esta radica en su capacidad para minimizar errores en modelos complejos, mejorando así la precisión general del aprendizaje automático. Además, se implementan algoritmos especializados para abordar problemas no convexos.

% EPI
La optimización de dos niveles es una herramienta clave en el diseño y operación de redes industriales sostenibles. Ejemplos notables incluyen redes de agua industrial, 
% Water integration in eco-industrial parks using a multi-leader-follower approach
donde en los estudios de \cite{Ramos2016WaterII} se optimizan mediante juegos de múltiples líderes-seguidores, priorizando objetivos ambientales y económicos. Los resultados muestran que las empresas participantes lograron beneficios significativos en escenarios con formulaciones Karush-Kuhn-Tucker (KKT). 
El enfoque MLFG (Multi-Leader-Follower Game) se utiliza para analizar la optimización de redes de agua en Entornos Industriales Productivos (EIP). En este contexto, las empresas buscan minimizar costos mientras que una autoridad regula el consumo de agua dulce. Se comparan diferentes formulaciones y métodos de solución, mostrando que MLFG es más confiable que la optimización multiobjetivo en escenarios multi-criterio. Además, se destaca la influencia de la estructura del juego en la configuración óptima, sugiriendo la necesidad de un diseño óptimo para cada planta dentro del EIP \cite{Ramos2016WaterII}. 
Los enfoques SLMFG y MLSFG presentan variaciones en el rol de los participantes: en SLMFG, las empresas son seguidoras y la autoridad es líder, mientras que en MLSFG, ocurre lo contrario. Se resalta que el enfoque MLFG logra equilibrar objetivos económicos y ambientales, generando ahorros significativos mediante la reutilización de recursos. Los resultados indican una reducción en el consumo de agua fresca gracias a las estrategias implementadas, utilizando herramientas como GAMS para modelar los problemas de optimización \cite{Ramos2016WaterII}. 
% Utility network optimization in eco-industrial parks by a multi-leader follower game methodology
Además, \cite{Ramos2018UtilityNO} introducen el concepto de autoridad ambiental en el diseño de redes de servicios públicos, utilizando juegos de múltiples líderes-seguidores y reformulaciones KKT. 
% Bi-level optimal low-carbon economic dispatch for an industrial park with consideration of multi-energy price incentive
En el ámbito del despacho energético bajo restricciones de carbono, \cite{Gu2020BilevelOL} modela incentivos de precios de energía en un parque industrial, demostrando que un enfoque binivel puede simultáneamente mejorar el impacto ambiental y los beneficios económicos, utilizando un procedimiento iterativo primal-dual

% SLSF 
%A subsidy policy to managing hazmat risk in railroad transportation network
Además investigaciones como las de \cite{Bhavsar2021ASP} sobre la aplicación de una política de subsidios para gestionar el riesgo de materiales peligrosos en una red de transporte ferroviario. En este modelo, el gobierno actúa como líder, ofreciendo subsidios para incentivar al operador ferroviario (el seguidor) a usar rutas alternativas que eviten los enlaces de alto riesgo en la red, utilizando el enfoque Single Leader Single Follower (SLSF). Los autores utilizan una reformulación de Karush-Kuhn-Tucker (KKT) para resolver el problema y aplican su método a un caso real en los Estados Unidos. Se demuestra que incluso subsidios modestos pueden resultar en una reducción significativa del riesgo.


% Bilevel Difíciles
% NP-Hard
% The polynomial hierarchy and a simple model for competitive analysis
% Some properties of the bilevel programming problem
Dado que los problemas de optimización de ese tipo son inherentemente difíciles de resolver debido a su naturaleza \textbf{NP-hard} \cite{Jeroslow1985ThePHNP,jonathan_f__bard_1991NP}
% Sigma P2Hard
% Libro Dempe
% Tesis doctoral Cerulli
o incluso $\Sigma P2-hard$ \cite{phdthesisCerulli,DempeyZemkoho2020},
se han desarrollado diversos enfoques para abordar su complejidad computacional. Entre los métodos exactos más utilizados se encuentran las reformulaciones basadas en las condiciones KKT, que permiten transformar el problema binivel en un problema mononivel resoluble mediante técnicas tradicionales de programación matemática. Sin embargo, estos enfoques suelen ser computacionalmente intensivos para problemas de gran escala \cite{phdthesisCerulli}.
%A Review on Bilevel Optimization: From Classical to Evolutionary Approaches and Applications
En paralelo, los algoritmos metaheurísticos, como los evolutivos, han ganado relevancia al proporcionar aproximaciones eficientes en casos no lineales o no convexos, donde las soluciones exactas son inalcanzables en tiempos razonables \cite{Sinha2017ARO}.
% Dividir en subproblemas y resolver iterativamente
% Collection of Test Problems for Constrained Global Optimization Algorithms
Otro enfoque destacado es el uso de métodos de descomposición, los cuales dividen el problema en subproblemas más manejables que pueden resolverse iterativamente \cite{Floudas1990ACO}.

% An SOS1-based approach for solving MPECs with a natural gas market application
Estas técnicas son particularmente útiles en aplicaciones prácticas, como los mercados de energía o los modelos de sostenibilidad \cite{SadddiquiNaturalGasSOS1}.
% Libro Dempe
A pesar de estos avances, existen desafíos abiertos. La escalabilidad sigue siendo un problema crítico, ya que el crecimiento exponencial de las opciones en problemas de gran tamaño limita la aplicabilidad de los métodos exactos \cite{DempeyZemkoho2020}. Asimismo, los problemas no convexos carecen de garantías de convergencia hacia el óptimo global, lo que los hace especialmente difíciles de abordar. Finalmente, la incorporación de incertidumbre en los modelos agrega una capa adicional de complejidad, lo que demanda nuevos enfoques híbridos que combinen algoritmos exactos y heurísticos para mejorar la eficiencia computacional sin sacrificar la calidad de las soluciones \cite{phdthesisCerulli,Sinha2017ARO}. Estos avances y desafíos reflejan la importancia de diseñar algoritmos personalizados que aprovechen las estructuras particulares de cada problema binivel. Las aplicaciones industriales, como el diseño de redes ecoindustriales y la gestión de mercados energéticos, destacan la necesidad de enfoques que equilibren precisión y tiempo de cálculo, haciendo de la optimización de dos niveles un área de investigación activa con un impacto significativo en la práctica.

%Aclarar que la interaccion jerarquica es compleja
En otras palabras, el problema de optimización binivel se centra en que el líder (nivel superior) debe tomar decisiones ($x$) que optimicen su objetivo $F(x, y)$, anticipando que el seguidor (nivel inferior) responderá de manera óptima con respecto a su propio objetivo $f(x, y)$, dado el valor de $x$ elegido por el líder. Esta interacción jerárquica entre ambos niveles añade una gran complejidad al problema en comparación con los problemas de optimización de un solo nivel.

% DIFERENCIA OPTIMISTA PESIMISTA
% Decir que en el libro de Dempe se hablan de estos dos enfoques
El concepto mismo de solución del problema bi-nivel es complejo. La decisión del líder es solo respecto a un grupo de variables, mientras que las otras influyen en la función objetivo, pero no son decisión de él. Si para un mismo valor de las variables del líder el problema del seguidor tiene diferentes soluciones óptimas, el valor de la función objetivo del líder no estará determinado, sino que depende de cuál de los óptimos escogió el otro agente. De ahí que se consideran dos enfoques principales: el optimista y el pesimista. En el enfoque optimista, se asume que el seguidor, que actúa en el nivel inferior, elegirá la solución más favorable para el líder, quien toma decisiones en el nivel superior. Este es considerado más tratable y, en ciertas situaciones favorables, puede simplificarse a un problema convexo. Además, en el contexto de múltiples objetivos, el enfoque optimista permite alcanzar el mejor frente de Pareto posible \cite{DempeyZemkoho2020}.

Por otro lado, el enfoque pesimista asume que el seguidor seleccionará la opción menos favorable para el líder entre las soluciones óptimas disponibles, el cual es más complejo de resolver y puede incluso no tener solución. A menudo, se requieren reformulaciones para abordar estos problemas, lo que lo convierte en un reto teórico y computacional significativo. Y en situaciones de múltiples objetivos, conduce al peor frente de Pareto posible \cite{Sinha2017ARO}.

Es relevante destacar que la mayoría de la literatura se centra en el enfoque optimista debido a su mayor facilidad de tratamiento. Sin embargo, el otro también tiene su utilidad, especialmente en la modelación de situaciones donde se considera la aversión al riesgo \cite{DempeyZemkoho2020}. En este contexto, los términos ''líder'' y ''seguidor'' se utilizan para describir los roles en el modelo a optimizar; el líder toma decisiones considerando las posibles reacciones del seguidor, quien a su vez reacciona seleccionando su mejor opción \cite{Sinha2017ARO}.

% Sobre Valor extremal KKT y algoritmos 
Para obtener las condiciones de optimalidad y los algoritmos de solución de los modelos binivel se reportan dos enfoques fundamentales. En \cite{DempeyZemkoho2020} se usa la función valor extremal. 
%Explicacion sobre que va el valor extremal
Esto significa que para todo valor de $x$, se considera la minimización de la función objetivo del líder en el conjunto dado por las restricciones de ambos individuos y la condición de que la función objetivo del seguidor es menor o igual que el valor más pequeño que alcanza en el conjunto de soluciones factibles del seguidor.
% Hablar sobre el enfoque KKT 
% Hablar sobre las particularidades de cuando es condicion necesaria y suficiente
Las condiciones KKT son una herramienta clave en la reformulación de problemas de optimización, particularmente cuando el problema de nivel inferior es convexo. 
% Condiciones necesarias y suficientes dictadas por el kkt
Estas condiciones son necesarias bajo regularidad del conjunto de soluciones factibles  y suficientes en problemas convexos 
como se establece en la literatura sobre programación no lineal.
 
% Al realizar el kkt en el level inferior se transforma a un MPEC
En \cite{Caselli2024BilevelOW,DempeyZemkoho2020,phdthesisCerulli} se habla de  sustituir el problema del nivel inferior por la condicion de KKT permite transformar problemas de optimización binivel en programas matemáticos con restricciones de equilibrio (MPEC), facilitando así su resolución.

% Flegel Kanzow Sobre MPECs y los ptos estacionarios
%En el contexto de los MPEC, en \cite{Flegel2003AFJ} se exponen varios tipos de puntos estacionarios que son cruciales para analizar la optimalidad. Un punto \textbf{débilmente estacionario} es aquel que satisface las condiciones básicas de equilibrio, siendo una condición necesaria pero no suficiente para la optimalidad local. La \textbf{C-estacionariedad} es una condición más fuerte, que además requiere que el producto de ciertos multiplicadores de Lagrange sea no negativo en el conjunto degenerado. A su vez, la \textbf{M-estacionariedad} es aún más restrictiva, ya que exige condiciones específicas sobre los multiplicadores en el conjunto degenerado (que o bien ambos sean positivos, o su producto sea cero). El artículo también introduce el concepto de \textbf{A-estacionariedad}, que surge del enfoque de Fritz John, donde se requiere que al menos uno de los multiplicadores sea no negativo en el conjunto degenerado. Finalmente, un punto es \textbf{fuertemente estacionario} si ambos multiplicadores son no negativos en el conjunto degenerado, siendo esta la condición más restrictiva y que se da bajo ciertas condiciones como MPEC-LICQ o MPEC-SMFCQ. Por ello, estas condiciones forman una jerarquía donde la M-estacionariedad implica la C-estacionariedad y esta a su vez, implica la estacionariedad débil, siendo la estacionariedad fuerte la más restrictiva de todas. \cite{Flegel2003AFJ}.

% Metodos matematicos que se utilizan KKT y branch and bround
Los algoritmos que se basan en las condiciones KKT incluyen una variedad de métodos, como técnicas de branch-and-bound y métodos de suavizado, ver \cite{DempeyZemkoho2020}. 
% Hablar sobre el Suavizado KKT y Del SQP ademas de que el KKT se extiende a algoritmos evolutivos
%TODO: Change
Estos algoritmos son utilizados para resolver problemas complejos de optimización que involucran restricciones. Por ejemplo, en \cite{DempeyZemkoho2020} se discute el uso de un método de suavizado junto con las condiciones KKT para abordar problemas relacionados con la optimización de hiperparámetros. Además, se menciona que los algoritmos SQP (Sequential Quadratic Programming) también se fundamentan en las condiciones KKT para resolver problemas suaves con restricciones. Donde enuncian que versatilidad del método KKT se extiende incluso a algoritmos evolutivos. Este hecho demuestra su versatilidad diversas áreas de la optimización.

% Que son no convexos por las transformaciones
Los problemas de optimización de dos niveles son inherentemente no convexos, lo que implica que los métodos de optimización convexa no son directamente aplicables. Esta no convexidad puede llevar a soluciones subóptimas y a dificultades para encontrar una solución global. Aunque en muchos casos las funciones y conjuntos factibles pueden ser convexos, la estructura general del problema sigue siendo no convexa.

%La dificultad para encontrar soluciones globales se ve exacerbada por la tendencia de los métodos de búsqueda local a quedar atrapados en óptimos locales. Por ello, se han desarrollado métodos de búsqueda global que consideran la estructura específica de los problemas de optimización de dos niveles. Estos enfoques suelen incluir fases tanto de búsqueda local como global, utilizando condiciones de optimalidad global para mejorar la efectividad del proceso \cite{DempeyZemkoho2020}.

%Además, para abordar la no unicidad de las soluciones y mejorar la estabilidad del sistema, se emplean métodos de regularización. Estos pueden implicar la regularización de la función objetivo del seguidor o del conjunto de respuestas óptimas del mismo \cite{DempeyZemkoho2020}.

En resumen, obtener garantías sobre soluciones en problemas de optimización de dos niveles es un desafío complejo debido a su no convexidad inherente, las dificultades para escapar de óptimos locales y la posible falta de unicidad y estabilidad en las soluciones. Los algoritmos frecuentemente dependen de puntos estacionarios, que no siempre corresponden a las soluciones óptimas deseadas. Por lo tanto, es crucial desarrollar métodos especializados que aborden estos problemas y permitan encontrar soluciones globales o aproximaciones adecuadas \cite{DempeyZemkoho2020}.

% Se añadio la genericidad
% Paper de gema sobte  los ptos estacionarios
En este contexto, se ha estudiado la estructura genérica de los problemas de complementariedad (MPCC) que surgen del enfoque KKT y Fritz-John (FJ) aplicado a un problema de dos niveles. Se ha demostrado que, para una clase amplia de estos, la condición de que independencia lineal (MPCC-LICQ) se cumple en todos los puntos factibles. Sin embargo, las condiciones de complementariedad estricta (MPCC-SC) y las condiciones de segundo orden (MPCC-SOC) pueden fallar en puntos críticos (estacionarios), incluso en situaciones genéricas \cite{Allende2012SolvingBP}. Esta situación complica aún más la obtención de garantías sobre la solución.

Es importante señalar que existen casos singulares donde los puntos estacionarios pueden ser problemáticos, especialmente cuando el multiplicador (\(\alpha\)) asociado a la condicion de KKT del problema de nivel inferior es igual a cero. En tales circunstancias, la condición MPCC-SC puede no cumplirse, lo que podría llevar a que el método KKT no funcione adecuadamente \cite{Allende2012SolvingBP}.

% Qué se va a hacer en la tesis
Basándose en la caracterización de los diferentes tipos de puntos estacionarios establecida por \cite{Flegel2003AFJ}, esta tesis propone desarrollar un generador de problemas que, dado un punto estacionario conocido y un conjunto de funciones que definen un problema de dos niveles, agregarles fucniones polinomicales de primer o segundo grado de forma tal que el punto inicial dado sea un punto critico del problema creado. 
Este generador facilitará el estudio del comportamiento de algoritmos conocidos en problemas con al menos un punto estacionario conocido.
El usuario además podrá decidir si quiere un punto crático con multiplicadores arbitrarios o si $\alpha = \vec{0}$, lográndose estudiar las clases de puntos críticos que aparecen en el caso genérico, o sea en un clase amplia y significativa de problemas

%NOTA:
% Clase significativa y amplia:
% Es que todo problema es límite de problemas de la clase,
% y si un problema está en la clase,
% para toda perturbación de clase C3 suficientemente pequeña,
% el problema perturbado sigue estando en la clase.


% Explicación 
% 2do cap
La tesis está compuesta de 3 capítulos, luego del capítulo de introducción, por un segundo capítulo 
que contiene la notación que se empleará, se define el problema de dos niveles con un líder y un seguidor, y se explica la teoría matemática para su transformación en un problema MPEC, así como los algoritmos de Julia que se utilizarán en ella.
% 3ero
En el tercer capítulo se explicará la implementación algorítmica propuesta anteriormente y su correcta utilización. 
% 4to
En el cuarto capítulo se analizarán los resultados obtenidos por el algoritmo propuesto y su comparación con algoritmos implementados en el entorno Julia.
% Siguiente
Finalmente, se presentarán las conclusiones y recomendaciones del trabajo realizado.
		
	\chapter{Preliminares}

La optimización binivel es un problema de optimización en el cual un subconjunto de variables está restringido a ser la solución óptima de otro problema de optimización, el cual está parametrizado por las variables restantes. Este tipo de problema tiene dos niveles jerárquicos de decisión: el problema de nivel superior o del líder, y el problema de nivel inferior o del seguidor. 
%Explicación breve de binivel
En términos abstractos, la optimización binivel busca minimizar una función objetivo de nivel superior, $F(x, y)$, donde $x$ son las variables de decisión del líder y $y$ son las variables del seguidor. Esta minimización está sujeta a dos tipos de restricciones: 
%Restricciones explicacion old
%las restricciones explícitas para el líder, $x \in X$, donde $X$ es el conjunto de valores factibles para las variables del líder; y las restricciones implícitas impuestas por el seguidor, donde $y$ debe pertenecer al conjunto de soluciones óptimas del problema de optimización del seguidor, $\arg\min\{f(x, y) : y \in Y(x)\}$. En este contexto, $f(x, y)$ es la función objetivo del nivel inferior, y $Y(x)$ representa las restricciones del nivel inferior, las cuales pueden depender de las variables de decisión del líder, $x$.
% Restricciones nuevas
\begin{itemize}
    \item \textbf{Restricciones explícitas:}
    \begin{itemize}
        \item Para el líder: $x \in X$, donde $X$ es el conjunto de valores factibles para las variables del líder.
    \end{itemize}
    
    \item \textbf{Restricciones implícitas:}
    \begin{itemize}
        \item Impuestas por el seguidor, donde $y$ debe pertenecer al conjunto de soluciones óptimas del problema de optimización del seguidor, $\arg\min\{f(x, y) : y \in Y(x)\}$.
        \item En este contexto, $f(x, y)$ es la función objetivo del seguidor, $Y(x)$ representa las restricciones del nivel inferior, las cuales pueden depender de las variables de decisión del líder, $x$.
    \end{itemize}
	\item Tambien se puede asumir que ambas variables tienen restricciones conjuntas, o sea que $(x,y) \in M^0$
	 
\end{itemize}


Un problema de optimización binivel tiene dos características principales: en primer lugar, el problema del nivel inferior actúa como una restricción para el problema del nivel superior, y en segundo lugar, la solución del nivel inferior depende del valor de las variables del nivel superior, creando una interdependencia entre ambos niveles. Por ello, el líder debe anticipar la respuesta óptima del seguidor al tomar sus decisiones.

La formulación general de un problema de optimización binivel se expresa matemáticamente como:  

% Definición de problema binivel
%\begin{equation}
%\begin{aligned}
%\text{minimizar} & \quad F(x, y) \\
%\text{sujeto a} & \quad G(x, y) \leq 0 \quad (\text{restricciones de desigualdad}) \\
%& \quad H(x, y) = 0 \quad (\text{restricciones de igualdad}) \\
%& \quad y \in S(x) = \arg \min_{y} \{ f(x, y) \mid V(x, y) \leq 0, U(x, y) = 0 \}.
%\end{aligned}
%\end{equation}


\textbf{Optimización bilevel optimista.} Un problema de optimización bilevel optimista, se formula como:

\begin{align}
    \min_{x \in X, y} & \quad F(x, y) \tag{1} \\
    \text{s.t.} & \quad G(x, y) \leq 0 \tag{2} \\
    & \quad y \in S(x), \tag{3}
\end{align}
Donde $S(x)$ es el conjunto de soluciones óptimas del problema parametrizado por $x$
\begin{align}
    \min_{y \in Y} & \quad f(x, y) \tag{4} \\
    \text{s.t.} & \quad V(x, y) \leq 0. \tag{5}
\end{align}
Donde $M(x,y)$ es el conjunto de restricciones comunes para ambos niveles

Donde $x \in R^{n}$, $y \in R^{m}$,  $F(x,y), f : \mathbb{R}^{n} \times \mathbb{R}^{m} \to \mathbb{R}$,  $g_i \in G(x,y)\| g_i(x,y) , f : \mathbb{R}^{n} \times \mathbb{R}^{m} \to \mathbb{R} $, 
$f(x,y), f : \mathbb{R}^{n} \times \mathbb{R}^{m} \to \mathbb{R}$,  $v_i \in V \| v_i(x,y), f : \mathbb{R}^{n} \times \mathbb{R}^{m} \to \mathbb{R}$,
$m_i \in M(x,y) \| m_i : f \mathbb{R}^{n} \times \mathbb{R}^{m} \to \mathbb{R}$

Los elementos clave en la optimización binivel incluyen las funciones objetivo $F(x, y)$ y $f(x, y)$, que corresponden a los objetivos del líder y del seguidor, respectivamente; las restricciones $G(x, y) \leq 0$ y $H(x, y) = 0$, que deben ser satisfechas por ambas partes; y el conjunto de soluciones del seguidor $S(x)$, el cual representa las soluciones óptimas del nivel inferior en función de las decisiones del líder.

\section{Transformación de los problemas de dos niveles}
		
		Los problemas de dos niveles pueden ser reformulados en un problema de un solo nivel al reemplazar el problema del nivel inferior por las condiciones KKT de este en las restricciones del primer nivel. 
		
		Para el caso de los SLSMG donde se tiene un problema de optimización en el nivel inferior este sustituye por de las condiciones KKT de este, obteniendo un MPEC \autocite{aussel2020}.
		
        \begin{table}
				Nota:
        	\begin{itemize}
        	    \item Sea $ J_0^v=\{j | V_i(x,y)=0\}$ al conjunto restricciones de desigualdad que sean índices activos
			
        	\end{itemize}
		\end{table}
        
% Descripcion del modelo en KKT
		\begin{table}[H]

		\[\begin{array}{l}
			\underset{\substack{x, y, \lambda_i}}{\min} \quad F(x, y)\\
			s.a \left\{ \begin{array}{l}
				
				g(x, y) \leq 0\\
				\nabla_{y} f(x, y) + \sum_{i=1}^{|J_{0}|} \nabla_{y} V_i(x, y) \lambda_i = 0 \\
				V_i(x, y) \leq 0 \\
				V_i(x, y)\lambda_i = 0 \\
				\lambda_i \geq 0\\
			\end{array}\right.
		\end{array}\]
		\caption*{MPEC resultante}
		\end{table}
% Añadir aclaratoria
Los tres últimos grupos de restricciones expresan que \( v \) y \( \lambda \) están restringidas en signo y que al menos una es 0. Estas condiciones son conocidas como **restricciones de complementariedad**. Estos modelos corresponden a la clase de problemas de programación matemática con restricciones de complementariedad (MPEC). A continuación, presentamos los resultados de esta área necesarios para el desarrollo de esta tesis.


% Modelación en Julia
% Se explica de que va BilevelJump
\section{Modelación en Julia}
% Que es BilvelJump
BilevelJuMP.jl es un paquete de Julia diseñado para modelar y resolver problemas de \textbf{optimización bilevel}, también conocidos como problemas de optimización de dos niveles o jerárquica. Estos problemas se caracterizan por tener dos niveles de decisión: un nivel superior y un nivel inferior, donde las decisiones del nivel superior influyen en las decisiones del nivel inferior, y viceversa \cite{BilevelJump}.
% Que problemas resuelve
Este paquete permite abordar una amplia variedad de tipos de problemas, incluyendo:

\begin{itemize}
    \item \textbf{Problemas de optimización bilevel generales:} BilevelJuMP.jl facilita la modelación de problemas que pueden representarse en la sintaxis de JuMP, ver documentación en \cite{JuMPPaper}, incluyendo restricciones lineales y no lineales, variables continuas y enteras, y diferentes tipos de objetivos.
    
    \item \textbf{Problemas con restricciones cónicas en el nivel inferior:} Maneja problemas donde el nivel inferior tiene una estructura de \textbf{programación cónica}, definiendo restricciones a través de conos convexos. Esto es útil para aplicar condiciones KKT en la reformulación del problema como MPEC.
    
    \item \textbf{Problemas con restricciones variadas en el nivel superior:} Permite gestionar una variedad de restricciones compatibles con JuMP, tales como restricciones cónicas, cuadráticas, no lineales y enteras.
    
    \item \textbf{Problemas con restricciones de equilibrio:} Facilita la transformación de problemas bilevel en problemas de programación matemática con restricciones de equilibrio (MPEC) y ofrece métodos para abordar las restricciones de complementariedad que surgen en estos casos.
    
    \item \textbf{Problemas con variables duales del nivel inferior en el nivel superior:} Permite la utilización de variables duales del nivel inferior explícitamente como variables en el nivel superior, lo cual es crucial para modelar problemas como la fijación de precios en mercados de energía.
    
    \item \textbf{Problemas con diferentes tipos de reformulaciones:} Los usuarios pueden experimentar con diversas reformulaciones para las restricciones de complementariedad en los problemas MPEC, incluyendo SOS1, restricciones de indicador, Fortuny-Amat y McCarl (Big-M), entre otros.
    
    \item \textbf{Problemas que requieren solvers MIP y NLP:} BilevelJuMP.jl puede utilizar tanto solucionadores de \textbf{programación lineal mixta entera (MIP)} como solucionadores de \textbf{programación no lineal (NLP)}, dependiendo de las características del problema y la reformulación elegida.
\end{itemize}

\subsection{Limitaciones del BilevelJuMP.jl}

A pesar de sus capacidades, BilevelJuMP.jl presenta algunas limitaciones, entre las cuales se encuentran:

\begin{itemize}
    \item Puede enfrentar dificultades en problemas altamente no lineales o con estructuras de optimización complejas que no se puedan representar adecuadamente en la sintaxis de JuMP.
    \item Existen ciertas restricciones que podrían no ser compatibles o que requieren transformaciones adicionales que podrían complicar el modelo.
    \item La eficiencia del paquete puede verse afectada en problemas de gran escala, donde el rendimiento del solver puede ser un factor crítico.
    \item La formulación y resolución de problemas muy específicos o especializados podrían no estar completamente optimizadas en el paquete.
\end{itemize}


% Explicacion algoritmos 
\section{Métodos de Reformulación para Optimización Bilevel}

La optimización bilevel presenta desafíos particulares debido a su naturaleza jerárquica y las condiciones de complementariedad resultantes. A continuación, se presentan los principales métodos de reformulación implementados en la literatura \cite{BilevelJump}.

\subsection{Método Big-M}

El método Big-M es una técnica fundamental para reformular problemas de optimización bilevel en problemas de programación matemática con restricciones de equilibrio (MPEC). Este método aborda específicamente las condiciones de complementariedad que surgen en estas reformulaciones, transformando el problema original en un problema de programación lineal mixta entera (MILP).

La reformulación mediante Big-M introduce un parámetro M suficientemente grande y variables binarias para transformar las condiciones de complementariedad no lineales en restricciones lineales. Para una condición de complementariedad de la forma $y_i(A_ix + b_i + D_iz) = 0$, el método introduce una variable binaria $f$ y las siguientes restricciones:

\begin{align*}
A_ix + b_i + D_iz &\leq M_p f \\
y_i &\leq M_d(1 - f) \\
f &\in \{0,1\}
\end{align*}

donde $M_p$ y $M_d$ son valores grandes para las variables primales y duales, respectivamente. La efectividad del método depende crucialmente de la selección apropiada de estos valores, que deben ser suficientemente grandes para no excluir la solución óptima, pero no excesivamente grandes para evitar inestabilidades numéricas \cite{BilevelJump}.

\subsection{Método SOS1}

El método de Conjuntos Ordenados Especiales de tipo 1 (SOS1) representa una alternativa más robusta para reformular las condiciones de complementariedad de Karush-Kuhn-Tucker (KKT) en problemas de optimización bilevel. A diferencia del método Big-M, SOS1 no requiere la determinación de parámetros adicionales, lo que lo hace particularmente atractivo en la práctica.

La reformulación SOS1 transforma una condición de complementariedad en una restricción que especifica que, como máximo, una variable en un conjunto puede tener un valor diferente de cero. Para una condición de complementariedad $y_i(A_ix + b_i + D_iz) = 0$, la reformulación SOS1 se expresa como:

\[ [y_i ; A_ix + b_i + D_iz] \in \text{SOS1} \]

Esta formulación está disponible en muchos solucionadores MILP modernos y ha demostrado un rendimiento competitivo \cite{BilevelJump}.

\subsection{Método ProductMode}

El método ProductMode representa un enfoque directo para manejar las condiciones de complementariedad en su forma de producto original. Este método es particularmente útil cuando se trabaja con solucionadores de programación no lineal (NLP), aunque no garantiza la optimalidad global.

La implementación del ProductMode mantiene la restricción de complementariedad en su forma original:

\[ y_i(A_ix + b_i + D_iz) \leq t \]

donde $t$ es un parámetro de regularización pequeño. Esta formulación, aunque no satisface las condiciones de calificación de restricciones estándar, es útil para obtener soluciones iniciales y puede ser especialmente efectiva cuando se combina con solucionadores NLP \cite{BilevelJump}.

\section{Puntos estacionarios en MPECs}
% Flegel Kanzow Sobre MPECs y los ptos estacionarios
En el contexto de los MPEC, en \cite{Flegel2003AFJ} se exponen varios tipos de puntos estacionarios que son cruciales para analizar la optimalidad. Un punto \textbf{débilmente estacionario} es aquel que satisface las condiciones básicas de equilibrio, siendo una condición necesaria pero no suficiente para la optimalidad local. La \textbf{C-estacionariedad} es una condición más fuerte, que además requiere que el producto de ciertos multiplicadores de Lagrange sea no negativo en el conjunto degenerado. A su vez, la \textbf{M-estacionariedad} es aún más restrictiva, ya que exige condiciones específicas sobre los multiplicadores en el conjunto degenerado (que o bien ambos sean positivos, o su producto sea cero). El artículo también introduce el concepto de \textbf{A-estacionariedad}, que surge del enfoque de Fritz John, donde se requiere que al menos uno de los multiplicadores sea no negativo en el conjunto degenerado. Finalmente, un punto es \textbf{fuertemente estacionario} si ambos multiplicadores son no negativos en el conjunto degenerado, siendo esta la condición más restrictiva y que se da bajo ciertas condiciones como MPEC-LICQ o MPEC-SMFCQ. Por ello, estas condiciones forman una jerarquía donde la M-estacionariedad implica la C-estacionariedad y esta a su vez, implica la estacionariedad débil, siendo la estacionariedad fuerte la más restrictiva de todas. \cite{Flegel2003AFJ}.


	\chapter{MODELACIÓN MATEMÁTICA E IMPLEMENTACIÓN}

Dado que por estudios previos (Nombrar )\\

Se tomará transformará el MPEC de la siguiente forma


\begin{table}[H]
    \begin{tabular}{l m{360pt}}
        $ F(x,y) $              & Función del líder                                                                                                          \\
        $ G(x,y) $              & Restricciones de desigualdad del líder                                                                                                            \\
        $ H(x,y) $                 & Restricciones de igualdad del lider                       \\
        $ f(x,y) $           & Funcion del follower                                                               \\
        $ V(x,y) $              &  Restricciones de desigualdad del follower  \\
        $ U(x,y) $     & restricciones de igualdad del follower\\
        $ V_{j}^{\star} $    & Restriccion de disigualdad del follower activa despues de modificarse el problema            \\
        $ \alpha  $             & Vector de entrada del problema puede ser el vector nulo  dimension igual que Cant V activas                                                                                                        \\
        $ \beta $               & Multiplicador asociado al $V_{j}^{\star}$ en el nivel superior          \\
        $ \lambda $              & Multiplicador asociado al $V_{j}^{\star}$ en el nivel inferior\\
        $\gamma$                & Valor asociado a cada $V_{j}^{\star}$ con $\alpha=\vec{0}$\\
    \end{tabular}

    \end{table}
%Notas de J_0^v
        %\begin{table}
		%		Nota:
        %	\begin{itemize}
        %	    \item Sea $ J_0^v=\{j | V_i(x,y)=0\}$ al conjunto restricciones de desigualdad que sean índices activos
		%	
        %	\end{itemize}
		%\end{table}
Se propone un generador de problemas prueba que dado un problema de optimización binivel y un punto 
se pueda conocer la factibilidad de los algoritmos tradicionales en dicho puntos (Explicar lo de los puntos estacionarios y ptos dengenerados )
Con este se puede conocer para los casos en que $\alpha=0$ y $\alpha>0$ y ver el comportamiento en estos.


Para utilizar el m'etodo propuesto debe darse como valores de entrada 


Se debe determinar cuales indices activos sus $\lambda = 0$ y cuales son $\lambda>0$ con el fin de poder calcular el valor de $b_j$ el se hará el siguente procedimiento

\begin{table}[H]
	$\lambda =0$:\\
	$(\nabla_{y}V_j+b_j)\alpha=0 \quad \forall V_j \in J_0$ \\
	$\lambda>0$:\\
	$[(\nabla_{y}V_j+b_j)\alpha]-\gamma_j=0 \quad \forall V_j \in J_0$


\end{table}

Despues de realizado el calculo del $b_j$ se procede a modificar su indice activo correspondiente:

\begin{table}[H]
	$V_{j}^{\star}=V_{j}+(b_j\alpha)$
\end{table}

Después se realiza las adicciones necesarias de una constante a cada restriccion con el fin de que sea factible el punto seleccionado
esta suma de constante no modifica las propiedades de convexidad del problema dado que una suma de funciones lineales no afecta.\\

% Decir que se realiza el KKT del nivel inferior

Despues se procede a realizar el KKT del nivel inferior con las modificiaciones anteriores 
Se plantea las condiciones KKT de las funciones evaluadas en el punto dado:
\begin{table}[H]
	$\nabla_{y}f(x,y)+\sum_{j=0}^{|V \in J_0|}(\lambda_j\nabla_{y}V_j(x,y))+\vec{bf}=\vec{0}$
\caption*{KKT del problema del nivel inferior}
\end{table}


Una vez hallado el KKT del nivel inferior se procede a hallar el KKT del MPEC resultante de este KKT el cual es:

% MPEC all KKT 
\begin{table}[H]
	$\nabla_{xy}F(x,y)+\sum_{i=1}^{|G \in J_o|}(\mu_i\nabla_{xy}g(x,y))+[\nabla_{x,y}\nabla_{y}f(x,y)+\sum_{j=1}^{|V \in J_0|}\lambda_j\nabla_{xy}\nabla_{y}V_j(x,y)]+\sum_{j=1}^{|V \in J_0|}(\beta_j\nabla_{xy}V_j(x,y))+\vec{BF}=\vec{0}$
\caption*{KKT del MPEC}
\end{table}


	
	\chapter{Experimentación}
En este capítulo se experimentará tomando una serie de problemas, ver \cite{BolibTestProblems}, 
que sean: Lineales, Cuadráticos y No Convexos. En ella primeramente utilizaremos las bibliotecas de Julia
para obtener puntos que sean mínimos locales, despues añadir valores aleatorios a esos puntos y posteriormente 
generar problemas estaciorarios del tipo: Fuertemente, M y C . Posteriormente estos problemas modificados serán nuevamente
ejecutados por los algoritmos tradicionales de Julia para conocer su efectividad con respecto al valor de la función objetivo del nivel superior.


Para ello tomaremos 15 problemas de optimización binivel SLSF, estos serán problemas 
Lineales, Cuadráticos y No Convexos, dividios en 5 problemas por cada clasificación anterior.
Estos han sido extraidos de \cite{Floudas1999HandbookOT} para las dos primeras clasificaciones y \cite{BolibTestProblems} para la última.

\newpage
\section{Problemas Escogidos}
Se muestran los problemas escogidos para la experimentación.

%Tabla con los problemas escogidos
\begin{table}[h!]
\centering
\caption{Problemas Seleccionados}
\begin{tabular}{ | m{5cm} | m{5cm} | m{5cm} | }
  
  \hline
  \textbf{No Convexos} & \textbf{Lineales} & \textbf{Cuadráticos} \\
  \hline
  MitsosBarton2006Ex312 & ex9.1.1 & ex9.2.1 \\
  \hline
  MitsosBarton2006Ex313 & ex9.1.2 & ex9.2.2 \\
  \hline
  MitsosBarton2006Ex314 & ex9.1.8 & ex9.2.3\\
  \hline
  MitsosBarton2006Ex323 & ex9.1.9 & ex9.2.4\\
  \hline
  MorganPatrone2006a & ex9.1.10 & ex9.2.5 \\
  \hline
\end{tabular}
\end{table}



\newpage
\section{Modelación de la experimentación}
Se describirá el proceso de generar la experimentación. 
Todos los valores han sido redondeados por exceso a dos cifras después de la coma. 

\subsubsection{Obtención de los óptimos}
Inicialmente se necesita obtener óptimos de los problemas con los paquetes convencionales de Julia
cuyos pasos son los siguientes:
\begin{itemize}
    \item \textbf{Problemas Lineales y Cuadráticos :}\\
            Con dicho problema se introduce los datos en la interfaz de \textbf{BilevelJuMP}, ver \cite{BilevelJump}, con el cual se utilizan
            Con ellas utilizamos 3 técnicas entre las ofrecidas por esta:
            \begin{itemize}
                \item \textbf{Big-M :} Con el optimizador High-Performance Solver for Linear Programming (HiGHS) y los valores $\text{primal big M} = 100, \quad \text{dual big M} = 100$.
                \item \textbf{SOS1 :} Con el optimizador Solving Constraint Integer Programs (SCIP).
                \item \textbf{ProductMode :} Con el optimizador Interior Point Optimizer (Ipopt).
            \end{itemize} 
            Cada uno de los resultados de evaluar el problema en cada forma anterior se guarda en un formato \textit{.xlsx}
            donde por cada optimizador se guarda los parámetros:
            \begin{itemize}
                \item Estatus del Primal, el cual define si es un punto factible o no.
                \item Estatus de la Finalización, si terminó porque encontro un óptimo o se estancó en un óptimo local.
                \item Valor de la función objetivo del nivel superior.
                \item El punto óptimo encontrado, en caso de ser hallado.
            \end{itemize} 
            Posteriormente se analizan los resultados de dichos métodos, se selecciona el de mejor evaluación de la función objetivo.
    \item \textbf{Problemas No Convexos :}\\
            Con dicho problema al \textbf{BilevelJuMP} no contar con soporte para esta clase de problemas binivel se utiliza \textbf{JuMP}, ver \cite{JuMPPaper}, 
            por ello se utiliza la reformulación KKT como la de \refeq{eq:KKT_Optimista} y se procede a utilizar la interfaz brindada por este, para el caso de las restricciones de 
            complementariedad se utiliza \textbf{Complementarity}, ver \cite{Complementarityjl}, con el optimizador Ipopt y análogo al caso anterior se extraen los mismos datos.
\end{itemize}


\subsubsection{Generación de los problemas}
% Como se modifica el punto
Se toma el problema original de entrada y el punto obtenido en el paso anterior el cual en cada componente se le hace suma un
valor aleatorio entre $1e-10$ y $5$, siendo este modificado: $z^*_0$. 
% Que se generan 3 clases de problemas estacionarios
Y se generan los 3 problemas bajo los 3 tipos de estacionariedad descritos en cada uno, 
% Se toma \alpha=0 y \alpha!=0
además en cada uno se toma la opcion de $\vec{\alpha}=\vec{0}$ y $\vec{\alpha}\ne \vec{0}$.
Para los $\vec{\alpha}=\vec{0}$ se genera un vector aleatorio donde cada componente está entre $1e-10$ y $3$.
% Como se dividen los indices activos
Luego para los conjuntos de índices activos de las $v_{j}s$ se dividen en $1/2$ del tipo $J_1^v$ \refeq{J_0_lambda_0_level_inferior} y $1/4$ para los dos restantes.
% Explicar como se modifican los indices activos 
Con respecto a la selección de los multiplicadores $\beta_j$ y $\gamma_j$ se 
se generan valores aleatorios entre $1e-10$ y $10$ en caso que estos no tengan que ser $0$, 
para los casos en que haya más de una combinación de los multiplicadores con respecto a su igualdad a $0$
se toma un valor aleatorio generado por una distribución uniforme discreta.
% Como se guarda
Finalmente cada problema generado es guardado en un archivo \textit{xlsx} con la siguiente designación:
\textit{(nombre del problema)\_(Tipo de punto estacionario)(generator)\_alpha\_((non\_zero) si $\alpha \neq 0$ y (zero) si $\alpha = 0$).xlsx}.
Donde se guarda: 
\begin{itemize}
    \item Las expresiones de las funciones objetivo de ambos niveles y su valor evaluado en el punto.
    \item Las restricciones de ambos niveles con sus multiplicadores respectivos, el tipo de indice activo y la evaluación de dicha función en el punto.
    \item El punto $z^*_0$.
    \item El $\vec{bf}.$
    \item El $\vec{BF}$.
    \item El $\vec{\alpha}$.
\end{itemize}

\subsubsection{Comparación de los algoritmos de Julia}
Después de tener generados los problemas se utilizan los mismos métodos de Julia mencionados anteiormente
para obtener óptimos de cada problema generado y se elige como representante el que más haya superado su óptimo o en caso de no superar el que mayor distancia tenga con el valor objetivo inicial.

\subsection{Resultados:}
Se presentan los resultados seleccionados bajo los criterios expuestos anteriormente en las siguientes tablas que contienen:
% Explicacion de las tablas
\begin{itemize}
    \item El nombre del problema original desde el cual fue modificado para que fuese estacionario de la clase deseada. 
    \item El punto al cual se forzó ser estacionario de la clase requerida.
    \item La evaluación de la función objetivo del punto estacionario.
    \item El punto óptimo hallado por los algoritmos de Julia.
    \item La evaluación de la función objetivo del óptimo.
    \item Método seleccionado.
\end{itemize}

\subsubsection{Para $\alpha =0$} 
% Tabla con los problemas lineales escogidos
%- Lineales
%	- C-Estacionario: ex9.1.1
%	- Fuertemente: ex9.1.10
%	- M-Estacionario:ex9.1.8
%	- Alpha=0: ex9.1.8
\begin{resultstable}{Problemas Seleccionados para $alpha=0$}
    \resultrow{ex9.1.8}{}{}{}{}{}
    \resultrow{ex9.2.3}{(1.55,2.7,-5.1,-8.65)}{-10.25}{(0,0,-5.1,-10)}{-14.7}{Big-M}
    \resultrow{MitsosBarton2006Ex313}{}{}{}{}{}
    \end{resultstable}







	
	\chapter{Conclusiones}

%Explicar que se hizo
En esta tesis se desarrolló un algoritmo para la generación de problemas de optimización binivel con características específicas de estacionariedad en puntos determinados. El mismo tiene la capacidad de modificar problemas originales para garantizar la factibilidad y estacionariedad en puntos dados, aprovechando las capacidades del lenguaje de programación Julia, que destaca por su alto rendimiento computacional y sintaxis adaptada a problemas de optimización.

% Explicar como fue la experimentación
La experimentación se llevó a cabo sobre tres categorías fundamentales de problemas: lineales, cuadráticos y no convexos. El proceso experimental comenzó con la obtención de puntos mínimos utilizando bibliotecas establecidas como BilevelJuMP y JuMP. Posteriormente, se generaron problemas estacionarios mediante la adición de componentes aleatorias a estos puntos para cada clase de problema definida.

% Que se comparó
Los problemas modificados fueron sometidos nuevamente a algoritmos tradicionales implementados en Julia para evaluar su efectividad frente a los puntos estacionarios generados. Se realizó un análisis comparativo destacando los casos más relevantes de cada categoría de puntos estacionarios, considerando la evaluación de la función objetivo del nivel superior en el punto donde se garantizó la estacionariedad, contrastándola con los resultados obtenidos por las bibliotecas convencionales.

% Recomendaciones
Como resultado de la investigación realizada, se han identificado varias líneas de trabajo futuro que permitirían expandir y mejorar los resultados obtenidos. En primer lugar, se recomienda ampliar el alcance de la experimentación numérica para incluir una mayor diversidad de problemas de optimización binivel. Esta expansión permitiría validar la robustez y versatilidad del algoritmo propuesto en diferentes contextos y escenarios de aplicación.

En segunda instancia, se sugiere profundizar en la investigación sobre la implementación del algoritmo desarrollado como criterio de parada en nuevos métodos de optimización binivel. Esta línea de investigación podría contribuir significativamente al desarrollo de algoritmos más eficientes y confiables, mejorando la capacidad de detectar y verificar puntos estacionarios durante el proceso de optimización.

Finalmente, se propone el desarrollo de una interfaz gráfica más intuitiva y funcional que facilite la generación automática de puntos según el tipo de estacionariedad requerida. Esta mejora en la usabilidad del software permitiría que usuarios con diferentes niveles de experiencia puedan aprovechar las capacidades del algoritmo de manera más efectiva, ampliando así su aplicabilidad práctica en diversos campos de estudio.

	%\include{Recomendaciones}
	
	\backmatter
	\printbibliography

\end{document}