\begin{opinion}
    La optimización binivel es, efectivamente, un problema complejo. La clave es la dependencia jerárquica: una decisión en el nivel superior afecta las opciones disponibles en el nivel inferior, y la solución óptima del nivel inferior a su vez impacta la función objetivo del nivel superior. Esto crea un ciclo de retroalimentación que dificulta la búsqueda de la solución óptima global. La estrategia de transformar el problema binivel a un MPEC (Mathematical Program with Equilibrium Constraints o Problema de Programación Matemática con Restricciones de Equilibrio) es una técnica común. Estas restricciones de complementariedad son las que hacen que los MPEC sean difíciles de resolver. La Contribución esta tesis es la generación de problemas cuya transformación MPEC tiene un punto estacionario con ciertas propiedades importantes en desde un punto de vista teórico y algorítmico. Tener una forma de generar problemas con puntos estacionarios de cierta clase podría ser muy valioso para evaluar el rendimiento de diferentes algoritmos de resolución de MPECs en problemas con características específicas, comprender las propiedades de estos modelos explorando la relación entre la estructura del problema y las propiedades de sus soluciones estacionarias. En
 La capacidad de generar problemas MPEC con tipos específicos de puntos estacionarios es una innovación que podría ser muy útil para la comunidad investigadora. Cabe destacar que esto se logra adicionando funciones lineales con lo que propiedades como linealidad y convexidad de las funciones involucradas no se ven afectadas.
Para un estudiante de Computación desarrollar esta tesis no es simple. Tuvo que adentrarse en el difícil campo de la teoría de la optimización binivel para lograr una programación del generador, utilizar Julia y entender los algoritmos de solución que se aplican.
Francisco ha sido trabajador, meticuloso, interesado y ha logrado llevar esta tesis a feliz término con independencia. Ha enfrentado varios retos y ha logrado sortearlos usando lo aprendido. Espero esta experiencia le sea útil en su vida futura
\end{opinion}