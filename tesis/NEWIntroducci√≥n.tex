\chapter{Introducción}
%\section{Definición de problema de optimización binivel}
% Obtener una garantía de la solución es muy complejo (generalmente es estacionario y con puntos estacionarios).
La optimización de dos niveles, un área fundamental en la investigación operativa y la teoría de juegos, presenta desafíos significativos debido a su complejidad inherente. Este tipo de problemas se caracterizan por la interacción entre un líder y un seguidor, donde las decisiones del líder afectan las respuestas del seguidor. Uno de los aspectos más críticos de esta problemática es garantizar la existencia de soluciones óptimas, lo cual se ve complicado por la naturaleza no convexa del problema, incluso cuando las funciones y los conjuntos factibles son convexos. A menudo, los algoritmos utilizados en este contexto solo logran identificar puntos estacionarios o críticos, que no necesariamente representan soluciones locales o globales, ver \cite{DempeyZemkoho2020}.

% Definición de problema binivel
%\begin{equation}
%    \begin{aligned}
%    \text{minimizar} & \quad F(x, y) \\
%    \text{sujeto a} & \quad G(x, y) \geq 0 \quad (\text{restricciones de desigualdad}) \\
%    & \quad H(x, y) = 0 \quad (\text{restricciones de igualdad}) \\
%    & \quad y \in S(x) = \arg \min_{y} \{ f(x, y) \mid V(x, y) \geq 0, U(x, y) = 0 \}.
%    \end{aligned}
% \end{equation}

El modelo de dos niveles es:

\begin{equation*} 
    \begin{array}{l}
       \min_x \quad F(x, y)\\
        s.a \left\{ \begin{array}{l}
            x \in {\cal{T}}, \\
             y \in S(x) = \arg  \min_y \{ f(x, y) \quad s.a \quad y \in  {\cal{H}}(x) \},,\\
           ( x,y) \in {\cal{M}}^0, 
        \end{array}\right.
        \end{array} \end{equation*}

% Explicar dimensiones 
donde $
x \in R^{n}, \; y \in R^{m}$,  $F : \mathbb{R}^{n} \times \mathbb{R}^{m} \to \mathbb{R}, $
  $  {\cal{T} }\subseteq \mathbb{R}^n ,$ $ f : \mathbb{R}^{n} \times \mathbb{R}^{m} \to \mathbb{R} ,
    $ $ {\cal{H}}(x) \subseteq \mathbb{R}^m ,$ para todo $x\in  \mathbb{R}^{n}$ and 
$
   {\cal{ M}}^0 \subseteq \mathbb{R}^{n + m}$.
%Aclarar que la interacción jerárquica es compleja
En otras palabras, el problema de optimización binivel se centra en que el líder (nivel superior) debe tomar decisiones ($x$) que optimicen su objetivo $F(x, y)$, anticipando que el seguidor (nivel inferior) responderá de manera óptima con respecto a su propio objetivo $f(x, y)$, dado el valor de $x$ elegido por el líder. Esta interacción jerárquica entre ambos niveles añade una gran complejidad al problema en comparación con los problemas de optimización de un solo nivel.

%Sección de aplicaciones
%\section{Aplicaciones}
% Aplicaciones 
Los problemas de optimización de dos niveles son muy utilizados para modelar y analizar mercados eléctricos complejos, ofreciendo una perspectiva única sobre las interacciones estratégicas entre diversos agentes económicos.
% Mercado Eléctrico
% Electricity spot market with transmission losses
%En modelos como los que proponen en \cite{Aussel2013ElectricitySM} se tienen en cuenta las pérdidas de transmisión, esta contribución mejora significativamente la representación del sistema eléctrico, permitiendo un análisis más profundo del equilibrio estratégico mediante técnicas de \textbf{optimización}. Al considerar las pérdidas de transmisión, el modelo captura aspectos fundamentales de la distribución y comercialización de energía que anteriormente pasaban desapercibidos.
% Deregulated electricity markets with thermal losses and production bounds: models and optimality conditions
En el trabajo de  \cite{Aussel2016DeregulatedEM} se desarrolló un modelo innovador que aborda los mercados de electricidad desregulados. Su enfoque se distingue por incorporar restricciones de producción y pérdidas térmicas, lo que permite una modelización más precisa y realista. Mediante el uso de modelos binivel, los investigadores pueden explorar escenarios más complejos y representativos del funcionamiento real de los mercados energéticos.
% Nash equilibrium in a pay-as-bid electricity market Part 2 - best response of a producer Didier
Otro ejemplo en esta línea se encuentra en \cite{Aussel2017NashEI}, se profundiza en el análisis de mercados de electricidad de pago por oferta, explorando cómo un productor puede ajustar su estrategia considerando las acciones de sus competidores. El estudio destaca la aplicación de conceptos de equilibrio de Nash y técnicas de mejor respuesta, proporcionando una metodología sofisticada para optimizar la participación de un productor en el mercado.


% Machine Learning
Los modelos binivel también tiene aplicaciones fundamentales en la selección de hiperparámetros en aprendizaje automático, como lo demuestra el trabajo de \cite{DempeyZemkoho2020ML}. El capítulo 6 del libro aborda la optimización de hiperparámetros en problemas de clasificación y regresión, presentando algoritmos innovadores para manejar funciones objetivo no suaves y no convexas. La razón del uso de esta radica en su capacidad para minimizar errores en modelos complejos, mejorando así la precisión general del aprendizaje automático. Además, se implementan algoritmos especializados para abordar problemas no convexos.

% EPI
La optimización de dos niveles es una herramienta clave en el diseño y operación de redes industriales sostenibles. Ejemplos notables se incluyen
% Water integration in eco-industrial parks using a multi-leader-follower approach
en los estudios de \cite{Ramos2016WaterII} donde se optimiza el uso del agua a partir de modelar la situación mediante juegos de múltiples líderes-seguidores, priorizando objetivos ambientales y económicos. Los resultados mostraron que las empresas participantes lograron beneficios significativos con las formulaciones \textbf{KKT} (Karush-Kuhn-Tucker) del modelo \textbf{MLFG} (Multi-Leader-Follower Game) utilizado.
Además, en \cite{Ramos2016WaterII} se destaca la influencia de la estructura del juego en la configuración óptima, sugiriendo la necesidad de un diseño óptimo para cada planta dentro del EIP. 
Los enfoques \textbf{SLMFG} (Single-Leader-Multifollower Game) y \textbf{MLSFG} (Multi-Leader-Single-Follower Game) presentan variaciones en el rol de los participantes: en SLMFG, las empresas son seguidoras y la autoridad es líder, mientras que en MLSFG, ocurre lo contrario. Se resalta que el enfoque MLFG logra equilibrar objetivos económicos y ambientales, generando ahorros significativos mediante la reutilización de recursos. Los resultados indican una reducción en el consumo de agua fresca gracias a las estrategias implementadas, utilizando herramientas como GAMS para modelar los problemas de optimización, ver \cite{Ramos2016WaterII}. 
% Utility network optimization in eco-industrial parks by a multi-leader follower game methodology
Además, en \cite{Ramos2018UtilityNO} los autores introducen el concepto de autoridad ambiental en el diseño de redes de servicios públicos, utilizando juegos de múltiples líderes-seguidores y reformulaciones KKT. 
% Bi-level optimal low-carbon economic dispatch for an industrial park with consideration of multi-energy price incentive
En el ámbito del despacho energético bajo restricciones de carbono, en \cite{Gu2020BilevelOL} se modela incentivos de precios de energía en un parque industrial, demostrando que un enfoque binivel puede simultáneamente mejorar el impacto ambiental y los beneficios económicos, utilizando un procedimiento iterativo primal-dual.

% SLSF 
%A subsidy policy to managing hazmat risk in railroad transportation network
Además estudios como los de \cite{Bhavsar2021ASP} investigan sobre la aplicación de una política de subsidios para gestionar el riesgo de materiales peligrosos en una red de transporte ferroviario. En este modelo, el gobierno actúa como líder, ofreciendo subsidios para incentivar al operador ferroviario (el seguidor) a usar rutas alternativas que eviten los enlaces de alto riesgo en la red, utilizando el enfoque \textbf{SLSF} (Single-Leader-Single-Follower). Los autores utilizan una reformulación de KKT para resolver el problema y aplican su método a un caso real en los Estados Unidos. Se demuestra que incluso subsidios modestos pueden resultar en una reducción significativa del riesgo.


%\section{Dificultades teóricas y computacionales de los problemas binivel}
El estudio de los problemas de dos niveles es de interés de la comunidad científica no solo porque modelan las situaciones antes mencionadas, sino porque es complejo obtener propiedades de sus soluciones así como su cálculo numérico.
% Bilevel Difíciles
El concepto mismo de solución del problema binivel es complejo. La decisión del líder es solo respecto a un grupo de variables, mientras que las otras influyen en la función objetivo, pero no son decisión de él. Si para un mismo valor de las variables del líder el problema del seguidor tiene diferentes soluciones óptimas, el valor de la función objetivo del líder no estará determinado, sino que depende de cuál de los óptimos escogió el otro agente. 

%Condiciones Necesarias
% Sobre Valor extremal KKT y algoritmos 
Para obtener las condiciones de optimalidad y los algoritmos de solución de los modelos binivel se reportan dos enfoques fundamentales. En \cite{DempeyZemkoho2020} se usa la función valor extremal. 
%Explicación sobre que va el valor extremal
Esto significa que para todo valor de $x$, se considera la minimización de la función objetivo del líder en el conjunto dado por las restricciones de ambos individuos y la condición de que la función objetivo del seguidor es menor o igual que el valor más pequeño que alcanza en el conjunto de soluciones factibles del seguidor.
 
% Al realizar el kkt en el level inferior se transforma a un MPEC
Otro enfoque clásico consiste en sustituir el problema del nivel inferior por la condición de KKT, lo que permite transformar problemas de optimización binivel en programas matemáticos con restricciones de equilibrio (MPEC), facilitando así su resolución, ver \cite{AnnotatedBibliographyDempe,Caselli2024BilevelOW,DempeyZemkoho2020}.
Conocido como \textit{enfoque KKT} en la literatura, es una de las formas más utilizada para la resolución de problemas de dos niveles en la actualidad, ver \cite{Aussel2021GenericityAO}.


% NP-Hard
% The polynomial hierarchy and a simple model for competitive analysis
% Some properties of the bilevel programming problem
Dado que los problemas de optimización de ese tipo son inherentemente difíciles de resolver debido a su naturaleza \textbf{NP-hard}, ver \cite{Jeroslow1985ThePHNP,jonathan_f__bard_1991NP}
% Sigma P2Hard
% Libro Dempe
% Tesis doctoral Cerulli
o incluso $\Sigma P2-hard$, ver \cite{phdthesisCerulli, DempeyZemkoho2020},
se han desarrollado diversos enfoques para abordar su complejidad computacional. 
Sin embargo, estos enfoques suelen ser computacionalmente intensivos para problemas de gran escala, ver \cite{phdthesisCerulli}.
%A Review on Bilevel Optimization: From Classical to Evolutionary Approaches and Applications
En paralelo, los algoritmos metaheurísticos, como los evolutivos, han ganado relevancia al proporcionar aproximaciones eficientes en casos no lineales o no convexos, donde las soluciones exactas son inalcanzables en tiempos razonables, ver \cite{Sinha2017ARO}.
% Dividir en subproblemas y resolver iterativamente
% Collection of Test Problems for Constrained Global Optimization Algorithms
Otro enfoque destacado es el uso de métodos de descomposición, los cuales dividen el problema en subproblemas más manejables que pueden resolverse iterativamente, ver \cite{Floudas1990ACO}.

% An SOS1-based approach for solving MPECs with a natural gas market application
Estas técnicas son particularmente útiles en aplicaciones prácticas, como los mercados de energía o los modelos de sostenibilidad, ver \cite{SadddiquiNaturalGasSOS1}.
% Libro Dempe
A pesar de estos avances, existen desafíos abiertos. La escalabilidad sigue siendo un problema crítico, ya que el crecimiento exponencial de las opciones en problemas de gran tamaño limita la aplicabilidad de los métodos exactos, ver \cite{DempeyZemkoho2020}. Asimismo, los problemas no convexos carecen de garantías de convergencia hacia el óptimo global, lo que los hace especialmente difíciles de abordar. Finalmente, la incorporación de incertidumbre en los modelos agrega una capa adicional de complejidad, lo que demanda nuevos enfoques híbridos que combinen algoritmos exactos y heurísticos para mejorar la eficiencia computacional sin sacrificar la calidad de las soluciones, ver \cite{phdthesisCerulli,Sinha2017ARO}. Estos avances y desafíos reflejan la importancia de diseñar algoritmos personalizados que aprovechen las estructuras particulares de cada problema binivel. Las aplicaciones industriales, como el diseño de redes ecoindustriales y la gestión de mercados energéticos, destacan la necesidad de enfoques que equilibren precisión y tiempo de cálculo, haciendo de la optimización de dos niveles un área de investigación activa con un impacto significativo en la práctica.



% Métodos matemáticos que se utilizan KKT y branch and bround
Los algoritmos que se basan en las condiciones KKT incluyen una variedad de métodos, como técnicas de branch-and-bound y métodos de suavizado, ver \cite{DempeyZemkoho2020}. 
% Hablar sobre el Suavizado KKT y Del SQP ademas de que el KKT se extiende a algoritmos evolutivos
Además, se emplean algoritmos SQP (Sequential Quadratic Programming) para resolver problemas suaves con restricciones al que se le aplica el enfoque KKT.  
% Que son no convexos por las transformaciones
Los problemas de optimización de dos niveles y la formulación KKT son inherentemente no convexos, lo que implica que los métodos de optimización convexa no son directamente aplicables. Esta no convexidad puede llevar a soluciones subóptimas y a dificultades para encontrar una solución global. 
Aunque en muchos casos las funciones involucradas en la definición del modelo sean convexas, la estructura general del problema sigue siendo no convexa.

En resumen, obtener garantías sobre soluciones en problemas de optimización de dos niveles es un desafío complejo debido a que por su no convexidad inherente, se presentan dificultades para escapar de óptimos locales, la posible falta de unicidad y la inestabilidad en las soluciones. Los algoritmos frecuentemente hallan puntos estacionarios, que no siempre corresponden a las soluciones óptimas deseadas. Por lo tanto, es crucial desarrollar métodos especializados que aborden estos problemas y permitan encontrar soluciones globales o aproximaciones adecuadas, ver \cite{DempeyZemkoho2020}.

% Se añadió la genericidad
% Paper de gema sobre  los puntos estacionarios
En este contexto, se ha estudiado la estructura genérica de los problemas de complementariedad (MPCC) que surgen del enfoque KKT y Fritz-John (FJ) aplicado a un problema de dos niveles. Se ha demostrado que, para una clase amplia de estos, la condición de que independencia lineal (MPCC-LICQ) se cumple en todos los puntos factibles. Sin embargo, las condiciones de complementariedad estricta (MPCC-SC) y las condiciones de segundo orden (MPCC-SOC) pueden fallar en puntos críticos (estacionarios), incluso en situaciones genéricas, ver \cite{Allende2012SolvingBP}. Esta situación complica aún más la obtención de garantías sobre la solución.

Es importante señalar que existen casos singulares donde los puntos estacionarios pueden ser problemáticos, especialmente cuando el multiplicador (\(\alpha\)) asociado a la condición de KKT del problema de nivel inferior es igual a cero. En tales circunstancias, la condición MPCC-SC puede no cumplirse, lo que podría llevar a que el método KKT no funcione adecuadamente, ver \cite{Allende2012SolvingBP}.

% Qué se va a hacer en la tesis
Basándose en la caracterización de los diferentes tipos de puntos estacionarios establecida por \cite{Flegel2003AFJ}, esta tesis propone desarrollar un generador de problemas que, dado un punto y las funciones  que definen un problema de dos niveles, agregarles funciones polinomiales de primer o segundo grado de forma tal que el punto inicial dado sea un punto crítico del problema creado. 
Este generador facilitará el estudio del comportamiento de algoritmos conocidos en problemas con ahora al menos un punto estacionario conocido.
El usuario además podrá decidir si quiere un punto crítico con multiplicadores arbitrarios o si $\alpha = \vec{0}$, lográndose estudiar las clases de puntos críticos que aparecen en el caso genérico, o sea en un clase amplia y significativa de los problemas generados.

%NOTA:
% Clase significativa y amplia:
% Es que todo problema es límite de problemas de la clase,
% y si un problema está en la clase,
% para toda perturbación de clase C3 suficientemente pequeña,
% el problema perturbado sigue estando en la clase.


% Explicación 
% 2do cap
La tesis está compuesta de 3 capítulos. Luego del capítulo de introducción,  
se mostrará la notación que se empleará, se define el problema de dos niveles con un líder y un seguidor, y se explica la teoría matemática para su transformación en un problema MPEC, así como los algoritmos de Julia que se utilizarán en ella.
% 3ero
En el tercer capítulo se explicará la implementación algorítmica propuesta anteriormente y su correcta utilización. 
% 4to
En el cuarto capítulo se analizarán los resultados obtenidos por el algoritmo propuesto y su comparación con algoritmos implementados en el entorno Julia.
% Siguiente
Finalmente, se presentarán las conclusiones y recomendaciones del trabajo realizado.